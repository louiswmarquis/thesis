% From mitthesis package
% Version: 1.07, 2024/09/26
% Documentation: https://ctan.org/pkg/mitthesis


\chapter{Convex Optimization of Decomposition of Electron-Electron Interaction Hamiltonian Term}

\section{Sparse Double Factorizations}

We would like to identify Hamiltonians for which the double factorization has ``easy'' rotations; that is, each $U_R(u^{(r)})$ decomposes to a small number of basis rotations.

Consider the minimum case: for all $r \in [0, n^2)$, $U_R(u^{(r)}) = I$, which occurs when $u^{(r)} = I$. That is, $Q^{(r)}$ is already diagonal with $\lambda'^{(r)}_s = Q^{(r)}_{s, s}$ and zero basis rotations are necessary. In this case, the Hamiltonian $H_{2e}$ itself can actually be written as a sum of products of number operators. 

\begin{equation}
    \begin{split}
        H_{2e} &= \frac{1}{2}\sum_r\lambda_r\left(\sum_{i,j} Q^{(r)}_{i, j}a^\dag_ia_j\right)^2 \\
        &= \frac{1}{2}\sum_r\lambda_r(\sum_{s} \lambda'^{(r)}_sn_s)^2 \\
        &= \frac{1}{2}\sum_{s, s'} n_sn_{s'}\sum_r\lambda_r\lambda'^{(r)}_s\lambda'^{(r)}_{s'} \\
        &= \frac{1}{2}\sum_s n_s\sum_r\lambda_r(\lambda'^{(r)}_s)^2 + \sum_{s < s'} n_sn_{s'}\sum_r\lambda_r\lambda'^{(r)}_s\lambda'^{(r)}_{s'} \label{eq: collapse}
    \end{split}
\end{equation}

With $H_{2e}$ in the form \eqref{eq: collapse} analogous to \eqref{eq: expansion} in the expansion method, $U_2 = e^{-iH_{2e}t}$ corresponds to a product of $n$ $R_z$ gates and ${n \choose 2}$ $CR_z$ gates, which, just like in the expansion method, costs $O(n^2\log_2(\frac{n}{\epsilon}))$ $T$ gates. However, unlike in the expansion method, these $O(n^2\log_2(\frac{n}{\epsilon}))$ gates suffice to simulate the entire $H_{2e}$, not just a term $H_2^{(r)}$. If the entire $H_{2e}$ can be simulated with $O(n^2\log_2(\frac{n}{\epsilon}))$ gates, it is unlikely that double factorization (and the coherent method) will be useful at all. The former produces one term that can be simulated with $O(n^2\log_2(\frac{n}{\epsilon}))$ $T$ gates. The latter produces $O(n^2)$ terms $H_2^{(r)}$ (requiring many Trotter or qDRIFT iterations), where each term can be simulated with $O(n\log_2(\frac{n}{\epsilon}) + \log_2(\frac{1}{\epsilon})^2)$ $T$-count (including a large constant factor).

More generally, if $Q{(r)}$ are mostly diagonally dominated,

More generally, if the $Q^{(r)}$ are generally very sparse, then the expansion of $H_{2e}$ also contains a reduced number of terms that make the coherent method unnecessary.

\begin{equation}
    \begin{split}
        H_{2e} &= \frac{1}{2}\sum_r\lambda_r\left(\sum_{i,j} Q^{(r)}_{i, j}a^\dag_ia_j\right)^2 \\
        &= \frac{1}{2}\sum_r\lambda_r(\sum_{s} \lambda'^{(r)}_sn_s)^2 \\
        &= \frac{1}{2}\sum_{s, s'} n_sn_{s'}\sum_r\lambda_r\lambda'^{(r)}_s\lambda'^{(r)}_{s'} \\
        &= \frac{1}{2}\sum_s n_s\sum_r\lambda_r(\lambda'^{(r)}_s)^2 + \sum_{s < s'} n_sn_{s'}\sum_r\lambda_r\lambda'^{(r)}_s\lambda'^{(r)}_{s'}
    \end{split}
    \label{eq: sparse}
\end{equation}

For the coherent method to be useful, all of the $U(u^{(r)})$ should decompose to $o(n^2)$ rotations, but $Q^{(r)}$ should have $\omega(\sqrt{n})$ non-zero terms (not a tight bound). This opens an area for future work to address: identifying useful Hamiltonians that satisfy these conditions.

In the next section, we will focus on one family of Hamiltonians that satisfy these conditions.

%We need it to be O(n) rotations but O(n^2) nonzeros in R.

%Example? k-local.

\section{K-Range Hamiltonian}

Define an electronic structure Hamiltonian from \eqref{eq: H3} as $K$-range if its values $h_{ijkl}$ satisfy the following condition.

\begin{equation}
    \begin{split}
        H &= \sum_{i, j} h_{ij}a^\dag_ia_j + \frac{1}{2}\sum_{i,j,k,l} h_{ijkl}a^\dag_ia^\dag_ja_ka_l  \\
        \text{such that } h_{ijkl} &= 0 \text{ if } \min(\max(|i - l|, |j - k|), \max(|i - k|, |j - l|)) > K
    \end{split}
\end{equation}

Any $K$-range Hamiltonian can be rewritten in the following more restrictive form. This is because, for an index $(i, j, k, l)$ such that $\max(|i - k|, |j - l|) > K$ but $\max(|i - l|, |j - k|) \leq K$, we can substitute $h_{ijkl}a^\dag_ia^\dag_ja_la_k = -h_{ijkl}a^\dag_ia^\dag_ja_ka_l$. This term satisfies the more restrictive condition and can be combined with the original term indexed $(i, j, l, k)$. From this process of substitutions and combination we get new terms $\bar{h}_{ijkl}$.

\begin{equation}
    \begin{split}
        H &= \sum_{i, j} h_{ij}a^\dag_ia_j + \frac{1}{2}\sum_{i,j,k,l} \bar{h}_{ijkl}a^\dag_ia^\dag_ja_ka_l \\
        \text{such that } \bar{h}_{ijkl} &= 0 \text{ if } \max(|i - l|, |j - k|) > K
    \end{split}
\end{equation}

Then, accordingly with \eqref{eq: H3}, this Hamiltonian can be rearranged, and accordingly with \eqref{eq: DF}, the second term can be factored. The condition does not change.

\begin{equation}
    \begin{split}
        H &= \sum_{i, j} h'_{ij}a^\dag_ia_j + \frac{1}{2}\sum_{i,j,k,l} h'_{iljk}a^\dag_ia_la^\dag_ja_k \\
        &= \sum_{i, j} h'_{ij}a^\dag_ia_j + \frac{1}{2}\sum_r\lambda_r\left(\sum_{i,j} Q^{(r)}_{i, j}a^\dag_ia_j\right)^2 \\
        \text{such that } h'_{iljk} &= \bar{h}_{ijkl} = 0 \text{ if } \max(|i - l|, |j - k|) > K
    \end{split}
\end{equation}

Observe that the condition implies that, for all $r$, $Q^{(r)}_{i, j} = 0$ if $|i - j| > K$. This means that all $Q^{(r)}$ are band matrices with bandwidth $K$.

\begin{equation}
    Q^{(r)} = \begin{bmatrix}
        Q^{(r)}_{0, 0} & \cdots & Q^{(r)}_{0, K} & 0 & \cdots & \cdots & \cdots & \cdots & 0 \\
        \vdots & \ddots & \vdots & Q^{(r)}_{1, K + 1} & \ddots & & & & \vdots \\
        Q^{(r)}_{K, 0} & \cdots & \ddots & \ddots & \ddots & \ddots & & & 0 \\
        0 & Q^{(r)}_{K + 1, 1} & \ddots & \ddots & \ddots & \ddots & \ddots & & 0 \\
        \vdots & \ddots & \ddots & \ddots & \ddots & \ddots & \ddots & \ddots & \vdots \\
        \vdots & & \ddots & \ddots & \ddots & \ddots & \ddots & Q^{(r)}_{n - K - 2, n - 2} & 0 \\
        \vdots & & & \ddots & \ddots & \ddots & \ddots & \cdots & Q^{(r)}_{n - K - 1, n - 1} \\
        \vdots & & & & \ddots & Q^{(r)}_{n - 2, n - K - 2} & \vdots & \ddots & \vdots \\
        0 & \cdots & \cdots & \cdots & \cdots & 0 & Q^{(r)}_{n - 1, n - K - 1} & \cdots & Q^{(r)}_{n - 1, n - 1}
    \end{bmatrix}
\end{equation}

If $Q^{(r)}$ has bandwidth $K$, this guarantees $u^{(r)}$ can be decomposed into at most $Kn$ Givens rotations, corresponding to at most $O(Kn)$ two-qubit basis rotations comprising $U(u^{(r)})$. \textcolor{red}{PROOF?} Meanwhile, $Q^{(r)}$ has $\Omega(Kn)$ nonzero terms, so the terms cannot combine in the way of \eqref{eq: collapse} and \eqref{eq: collapse2}. Therefore, a $K$-range Hamiltonian can be expressed as a sum of $n^2$ terms, each simulable with $O(n(\log{n} + K))$ gates using the coherent method. We see that the coherent method is therefore particularly useful on such low-range Hamiltonians. For example, if $K = O(\log{n})$, the total number of gates in a single Trotter iteration over these terms is $O(n^3\log{n})$, beating the $O(n^4)$ complexity of a brute force method.

\section{Decomposition of Hamiltonian Term}

We can expand our usage of this coherent method to electronic structure Hamiltonians that are ``low-rangedly dominant''. These are Hamiltonians that are not themselves low-range but can be expressed as a sum of a low-range term $H_0$ and a relatively small arbitrary-range term $\Delta H$.

\begin{equation}
    \begin{split}
        H &= H_0 + \Delta H \\
        ||\Delta H|| &\ll ||H_0|| \\
    \end{split}
    \label{eq: LRD}
\end{equation}

$e^{-iH_0t}$ is (asymptotically) cheaper, while $e^{-i\Delta Ht}$ is asymptotically more expensive but it affects the output to a lesser extent (since it's smaller). The idea, then, is to use a randomized Hamiltonian simulation algorithm such as qDRIFT, which samples terms more if they have a higher weight (operator norm). In simulating \eqref{eq: LRD}, the algorithm would perform $e^{-iH_0t}$ much more frequently (asymptotically) than $e^{-i\Delta Ht}$, which is good because the former is cheaper. If the operator norm discrepancy $||\Delta H|| \ll ||H_0||$ is large enough, $\Delta H$ will not contribute too much to the overall cost, while still being performed enough times to simulate $H$ precisely.


\section{Gradient Descent}
\section{Gate Count Analysis}
\section{Numerics}


%%%%%%%%%%%%%%%% end table %%%%%%%%%%%%%%%%%%% 

