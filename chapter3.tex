% From mitthesis package
% Version: 1.07, 2024/09/26
% Documentation: https://ctan.org/pkg/mitthesis


\chapter{Low Range Hamiltonians}

\section{Sparse Hamiltonians}

We would like to identify Hamiltonians for which the double factorization has ``easy'' rotations; that is, each $U_R(u^{(r)})$ decomposes to a small number of basis rotations.

Consider the minimum case: for all $r \in [0, n^2)$, $U_R(u^{(r)}) = I$, which occurs when $u^{(r)} = I$. That is, $Q^{(r)}$ is already diagonal with $\lambda'^{(r)}_s = Q^{(r)}_{s, s}$ and zero basis rotations are necessary. In this case, the Hamiltonian $H_{2e}$ itself can actually be written as a sum of products of number operators. 

\begin{equation}
    \begin{split}
        H_{2e} &= \frac{1}{2}\sum_r\lambda_r\left(\sum_{i,j} Q^{(r)}_{i, j}a^\dag_ia_j\right)^2 \\
        &= \frac{1}{2}\sum_r\lambda_r(\sum_{s} \lambda'^{(r)}_sn_s)^2 \\
        &= \frac{1}{2}\sum_{s, s'} n_sn_{s'}\sum_r\lambda_r\lambda'^{(r)}_s\lambda'^{(r)}_{s'} \\
        &= \frac{1}{2}\sum_s n_s\sum_r\lambda_r(\lambda'^{(r)}_s)^2 + \sum_{s < s'} n_sn_{s'}\sum_r\lambda_r\lambda'^{(r)}_s\lambda'^{(r)}_{s'} \label{eq: collapse}
    \end{split}
\end{equation}

With $H_{2e}$ in the form \eqref{eq: collapse} analogous to \eqref{eq: expansion} in the expansion method, $U_2 = e^{-iH_{2e}t}$ corresponds to a product of $n$ $R_z$ gates and ${n \choose 2}$ $CR_z$ gates, which, just like in the expansion method, costs $O(n^2\log_2(\frac{n}{\epsilon}))$ $T$ gates. Then Trotterization over $H_{2e}$ is not necessary because there is only one term. In such a case, it's unlikely that double factorization will be useful at all, since $H_{2e}$ is already so simple.

More generally, if $H_{2e}$ is very sparse, then a direct brute force Trotterization will not involve as many terms as the worst case $O(n^4)$, and its gate count may be low enough to make double factorization (and thereby the coherent method) unnecessary. At the same time, the coherent method requires the number of rotations composing $u^{(r)}$ to be small, and having very few rotations tends to make $Q^{(r)}$ (and thereby $H_{2e}$) very sparse. Therefore, this opens the problem of finding Hamiltonians $H_{2e}$ that are sufficiently dense but for which the $Q^{(r)}$ are sufficiently easily decomposable.

%We need it to be O(n) rotations but O(n^2) nonzeros in R.

%Example? k-local.

\section{K-Range Hamiltonian}

Define an electronic structure Hamiltonian from \eqref{eq: H3} as $K$-range if its values $h_{ijkl}$ satisfy the following condition.

\begin{equation}
    \begin{split}
        H &= \sum_{i, j} h_{ij}a^\dag_ia_j + \frac{1}{2}\sum_{i,j,k,l} h_{ijkl}a^\dag_ia^\dag_ja_ka_l  \\
        \text{such that } h_{ijkl} &= 0 \text{ if } \min(\max(|i - l|, |j - k|), \max(|i - k|, |j - l|)) > K
    \end{split}
\end{equation}

Any $K$-range Hamiltonian can be rewritten in the following more restrictive form. This is because, for an index $(i, j, k, l)$ such that $\max(|i - k|, |j - l|) > K$ but $\max(|i - l|, |j - k|) \leq K$, we can substitute $h_{ijkl}a^\dag_ia^\dag_ja_la_k = -h_{ijkl}a^\dag_ia^\dag_ja_ka_l$. This term satisfies the more restrictive condition and can be combined with the original term indexed $(i, j, l, k)$. From this process of substitutions and combination we get new terms $\bar{h}_{ijkl}$.

\begin{equation}
    \begin{split}
        H &= \sum_{i, j} h_{ij}a^\dag_ia_j + \frac{1}{2}\sum_{i,j,k,l} \bar{h}_{ijkl}a^\dag_ia^\dag_ja_ka_l \\
        \text{such that } \bar{h}_{ijkl} &= 0 \text{ if } \max(|i - l|, |j - k|) > K
    \end{split}
\end{equation}

Then, accordingly with \eqref{eq: H3}, this Hamiltonian can be rearranged into $H_{1e}$ and $H_{2e}$, and accordingly with \eqref{eq: DF}, $H_{2e}$ can be double factorized. The condition does not change.

\begin{equation}
    \begin{split}
        H_{2e} &= \frac{1}{2}\sum_{i,j,k,l} h'_{iljk}a^\dag_ia_la^\dag_ja_k \\
        &= \frac{1}{2}\sum_r\lambda_r\left(\sum_{i,j} Q^{(r)}_{i, j}a^\dag_ia_j\right)^2 \\
        &= \sum_{r = 0}^{R - 1} H_{2e}^{(r)} \\
        \text{such that } h'_{iljk} &= \bar{h}_{ijkl} = 0 \text{ if } \max(|i - l|, |j - k|) > K
    \end{split}
\end{equation}

Observe that fewer than $K^2n^2$ indices $iljk$ satisfy the range condition, so $H_{2e}$ can be expressed as the sum of $O(K^2n^2)$ Pauli tensor products in context of the brute force method. As one might expect, decreasing $K$ makes $H_{2e}$ sparser.

Observe also that the range condition implies, for all $r$, that $Q^{(r)}_{i, j} = 0$ if $|i - j| > K$. This means that all $Q^{(r)}$ are band matrices with bandwidth $K$.

\begin{equation}
    Q^{(r)} = \begin{bmatrix}
        Q^{(r)}_{0, 0} & \cdots & Q^{(r)}_{0, K} & 0 & \cdots & \cdots & \cdots & \cdots & 0 \\
        \vdots & \ddots & \vdots & Q^{(r)}_{1, K + 1} & \ddots & & & & \vdots \\
        Q^{(r)}_{K, 0} & \cdots & \ddots & \ddots & \ddots & \ddots & & & 0 \\
        0 & Q^{(r)}_{K + 1, 1} & \ddots & \ddots & \ddots & \ddots & \ddots & & 0 \\
        \vdots & \ddots & \ddots & \ddots & \ddots & \ddots & \ddots & \ddots & \vdots \\
        \vdots & & \ddots & \ddots & \ddots & \ddots & \ddots & Q^{(r)}_{n - K - 2, n - 2} & 0 \\
        \vdots & & & \ddots & \ddots & \ddots & \ddots & \cdots & Q^{(r)}_{n - K - 1, n - 1} \\
        \vdots & & & & \ddots & Q^{(r)}_{n - 2, n - K - 2} & \vdots & \ddots & \vdots \\
        0 & \cdots & \cdots & \cdots & \cdots & 0 & Q^{(r)}_{n - 1, n - K - 1} & \cdots & Q^{(r)}_{n - 1, n - 1}
    \end{bmatrix}
\end{equation}

If $Q^{(r)}$ has bandwidth $K$, this guarantees $u^{(r)}$ can be decomposed into fewer than $Kn$ Givens rotations, corresponding to at most $O(Kn)$ two-qubit basis rotations comprising $U(u^{(r)})$ \footnote{To see this, observe that (as quoted from \cite{FSN} and appropriated for our purposes), when a Givens rotation matrix $r_{pq}(\theta)$ left multiplies a matrix, it effects a rotation between rows $p$ and $q$ which can be used to zero out a single element in one of the rows. In \cite{FSN} there were ${n \choose 2}$ elements below the diagonal, in $Q^{(r)}$ there are ${n \choose 2} - {n - K \choose 2} = \frac{2Kn - K^2 - 2n + K}{2}< Kn$. So $u^{(r)}$ requires less than $Kn$ Givens rotations (on each side) to make $Q^{(r)}$ diagonal.}

Therefore, a $K$-range Hamiltonian can be expressed as a sum of $R$ terms $H_{2e}^{(r)}$, each simulable with either the $O(n^2)$ expansion method or the $O(n(\log{n} + K))$ $T$-gates (we ignore $\epsilon$ factors for simplicity).

\ref{table:1} lists bounds on second-order Trotterization $T$-counts using the brute force, double-factorized expansion, and double-factorized coherent methods. It considers their performance on generic $H_{2e}$ as well as ones that are $K$-range, linear-rank, or both. We chose to list second-order Trotterization costs rather than first-order to show that the double-factorized coherent method holds advantages over brute-force methods even in higher-order Trotterization, reduces the asymptotic impact of $L$, which is more present in the latter.

Observe that the coherent method has the lowest asymptotic upper bound among all three methods for low-rank $K$-range Hamiltonians, as long as $K = o(n)$ (otherwise it's the same as the expansion method). This remains the case ever for higher-order Trotterization. Of course, this is assuming that $K$ is not so low (i.e. $K = 0$) that $H_{2e}$ simplifies into a sum of commuting terms (which comprise one Trotter term) in the way of \eqref{eq: collapse}.
For generic-rank $K$-range $H_{2e}$, it performs asymptotically better than brute force if $K = \omega(n^{0.25})$ and better than expansion if $K = o(n)$. This also remains the case for arbitrary $k$th-order Trotterization but with a generic lower bound $K = \omega(n^{\frac{1}{3 + \frac{1}{k}}})$. For example, for fourth-order Trotterization ($k = 2$), we would need $K = \omega(n^{\frac{1}{3.5}})$.

\begin{table}[h!]
    \centering
    \renewcommand{\arraystretch}{1.25}
    \begin{tabular}{||c c c c c||} 
        \hline
         & $L$ & $\tilde{N}_T$ & $M = O(L^{1.5})$ & Total $= ML\tilde{N}_T$ \\
        \hline\hline
        Generic $H_{2e}$ & & & & \\
        \hline
        Brute Force & $O(n^4)$ & $O(1)$ & $O(n^6)$ & $O(n^{10})$ \\ 
        DF Expansion Method & $R = O(n^2)$ & $O(n^2)$ & $O(n^3)$ & $O(n^7)$ \\
        DF Coherent Method & $R = O(n^2)$ & $O(n^2)$ & $O(n^3)$ & $O(n^7)$ \\
        \hline\hline
        K-Range & & & & \\
        \hline
        Brute Force & $O(K^2n^2)$ & $O(1)$ & $O(K^3n^3)$ & $O(K^5n^5)$ \\ 
        DF Expansion Method & $R = O(n^2)$ & $O(n^2)$ & $O(n^3)$ & $O(n^7)$ \\
        DF Coherent Method & $R = O(n^2)$ & $O(Kn)$ & $O(n^3)$ & $O(Kn^6)$ \\
        \hline\hline
        Low Rank ($R = O(n)$) & & & & \\
        \hline
        Brute Force & $O(n^4)$ & $O(1)$ & $O(n^6)$ & $O(n^{10})$ \\ 
        DF Expansion Method & $R = O(n)$ & $O(n^2)$ & $O(n^{1.5})$ & $O(n^{5.5})$ \\
        DF Coherent Method & $R = O(n)$ & $O(n^2)$ & $O(n^{1.5})$ & $O(n^{5.5})$ \\
        \hline\hline
        Low Rank and K-Range & & & & \\
        \hline
        Brute Force & $O(K^2n^2)$ & $O(1)$ & $O(K^3n^3)$ & $O(K^5n^5)$ \\ 
        DF Expansion Method & $R = O(n)$ & $O(n^2)$ & $O(n^{1.5})$ & $O(n^{4.5})$ \\
        DF Coherent Method & $R = O(n)$ & $O(Kn)$ & $O(n^{1.5})$ & $O(Kn^{3.5})$ \\
        \hline
    \end{tabular}
    \caption{Asymptotic $T$-count of the brute force, double-factorized expansion method, and double-factorized coherent method on various types of Hamiltonians. We assume a second-order Trotterization. We ignore $\log_2{n}$ terms and $\log_2{\frac{1}{\epsilon}}$ terms for simplicity.}
    \label{table:1}
\end{table}

\section{Extraction of Low-Range Terms from Practical Electronic Structure Hamiltonians}

We have shown some asymptotic advantages for the coherent method within a deterministic Trotterization assumption. In practice, the large empirical discrepancy between the absolute values of the largest and smallest terms in typical electronic structure Hamiltonians makes qDRIFT much more suitable than deterministic Trotterization. In addition, electronic structure Hamiltonians typically are not strictly low-range, but often elements in high-range indices tend to be very small, and much of the ``weight'' is concentrated in low-range areas. We can combine these two concepts to potentially leverage the asymptotic advantage of the coherent method towards simulating generic electronic structure Hamiltonians that are ``low-range dominant''.

We propose a framework for decomposition in which $H_{2e}$ is broken into a sum of a term $H_L$ that has small rank $R$ and low range $K$ and a term $H_H$ that contains ``everything else'' and has no restrictions. Then we would qDRIFT over the $R$ terms in $H_L$ and the $O(n^4)$ tensor-product Pauli terms in $H_H$. Setting $R = 0$ would be equivalent to a conventional qDRIFT approach over the entire $H_{2e}$. The idea is that as $R$ is increased and weight is ``moved'' from $H_H$ to $H_L$, $H_L$ should gradually ``capture'' most of the weight in the Hamiltonian if it is indeed low-range dominant. Then, $H_H$ would be left with many very small terms, which can be handled effectively with qDRIFT, whose complexity scales only with the total weight, rather than the number of terms. Meanwhile, $H_L$ is able to handle the larger-valued low-range ``weight'' more efficiently due to its asymptotic advantages.

\begin{equation}
    \begin{split}
        H_{2e} &= \frac{1}{2}\sum_{i,j,k,l} h'_{iljk}a^\dag_ia_la^\dag_ja_k \\
        &= H_L + H_H
    \end{split}
\end{equation}

Unfortunately, it was not feasible to provide a specific implementation or numerical results in context of this framework, due to both the limited time frame of this project and the fact that $T$-count advantages of the coherent method only manifest asymptotically above a threshold at which Hamiltonian sizes become difficult for classical computers to manipulate. However, we hope that presenting this potential framework both conveys the applicability of the coherent method in realistic electronic structure Hamiltonians and inspires future work in this direction.


%%%%%%%%%%%%%%%% end table %%%%%%%%%%%%%%%%%%% 

