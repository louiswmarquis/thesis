% From mitthesis package
% Version: 1.01, 2023/06/19
% Documentation: https://ctan.org/pkg/mitthesis
%
% The abstract environment creates all the required headers and footnote. 
% You only need to add the text of the abstract itself.
%
% Approximately 500 words or less; try not to use formulas or special characters
% If you don't want an initial indentation, do \noindent at the start of the abstract

Quantum computational chemistry aims to use quantum computers to model molecular systems where classical computers struggle to handle the exponential complexity of quantum chemical effects. Typically, the simulation of such a physical system involves decomposing its Hamiltonian and approximating its evolution with deterministic Trotter-like methods or randomized algorithms such as qDRIFT. One important Hamiltonian is the electronic structure Hamiltonian arising from the Born-Oppenheimer approximation, which decomposes to $O(n^4)$ terms in the generic case, leading to very high asymptotic gate costs. A decomposition technique called double factorization allows an asymptotic reduction of the number of terms at the expense of an increased cost of implementing an individual term. We propose, design, and analyze in detail a quantum circuit that can reduce the asymptotic cost of such an individual term for certain classes of Hamiltonians. Specifically, for a ``$K$-range Hamiltonian'', a term that we define, we reduce the cost of a double-factorized term from $O(n^2)$ to $O(n(\log_2{n} + K))$. We also show that the orbital count threshold at which the proposed method's asymptotic advantages translate into numerical cost savings is fairly low, enabling its use in realistic chemical systems. We also propose a framework for expanding the application of this circuit from strictly low-range Hamiltonians to generic real-world Hamiltonians, which are often ``low-range dominant''.