% !TEX encoding = UTF-8 Unicode
% !BIB TS-program = biber

% This file is MIT-Thesis.tex, a LaTeX template for formatting an MIT thesis with the mitthesis class.
%
% Version: 1.17, 2024/11/02
%
% Author: John H. Lienhard, copyright 2024. Reuse under the MIT license: https://ctan.org/license/mit 

% Documentation is here: https://ctan.org/pkg/mitthesis

%% Don't modify the \DocumentMetadata command unless you know what it does. 
%% If this command throws an "undefined" error, your latex system is out of date: try commenting this command out.
\DocumentMetadata{
	lang		= en-US,
	pdfversion  = 1.7,
	pdfstandard = a-2b,
%	 pdfversion  = 2.0,
%    pdfstandard = a-4,
%	 debug		= {xmp-export}, % creates and xmpi file useful for checking metadata
}

%%%%%%%%%%%%%%%%%%%%%%%%%%%%%%%%%%%%%%%

\documentclass[twoside]{mitthesis}% fontset=newtx, fontset=libertine, fontset=libertinus, fontset=lmodern, fontset=newtx-sans-text, fontset=fira-newtxsf, fontset=heros-stix2, fontset=stix2, fontset=lmodern
%
% option [twoside]		gives facing-page behavior for printing; omitting twoside will eliminate even-numbered blank pages.
% option [lineno]	 	provides line numbers, as for editing
% option [mydesign] 	loads packages for color, title and list formats, margins, or captions: edit mydesign.tex to change defaults.
% option [fontset] is a keyvalue which can be:
%					 	for pdftex or unicode engines:  defaultfonts, libertine, libertinus, lmodern, lucida
%					 	for pdftex only: 				fira-newtxsf, newtx, newtx-sans-text
%						for unicode engines (luatex):	heros-stix2, stix2, termes, termes-stix2
%					 	if no key value is given, fonts default to CMR (pdftex) or LMR (unicode), i.e., "the LaTeX font".
%					 	You can edit the fontset files or you can write your own, myfonts.tex, and do [fontset=myfonts].
%						If you are using multiple languages, load the babel package in your fontset file, before the fonts.

%%%%%%%%% Packages used in sample chapters (not otherwise required) %%%%%%%

%% Package for code listing in Appendix A.
\usepackage{listings}%   documentation is here https://ctan.org/pkg/listings

%% Set chemical formulas nicely
\usepackage[version=4]{mhchem}%   documentation at https://ctan.org/pkg/mhchem

%% Latin filler used in Chapter 1, with a test for package version date (https://ctan.org/pkg/lipsum)
\usepackage{lipsum}
\IfPackageAtLeastTF{lipsum}{2021/09/20}{\setlipsum{auto-lang=false}}{}

%% Table related packages  

\usepackage{booktabs}% publication quality tables (https://ctan.org/pkg/booktabs)

\usepackage{array}% Additional options for column formats (https://ctan.org/pkg/array)

\usepackage{dcolumn}% For alignment of numbers on the decimal place (https://ctan.org/pkg/dcolumn) 
  \newcolumntype{d}[1]{D{.}{.}{#1}}% use with dcolumn package. Note: dcolumns are set in math mode.
  % syntax: d{x.y} where x is max number of digits before decimal and y is max number after.

% Package for multipage table in Appendix B.
\usepackage{longtable}% typeset multi-page tables (https://ctan.org/pkg/longtable)

%\usepackage{tabularx}% adjustable-width columns in tabular (https://ctan.org/pkg/tabularx)


%% Package for improved typography

\usepackage{microtype}% typographic fine-tuning, used in sample thesis committee page, but also acting globally on the text 

\usepackage[T1]{fontenc}
\usepackage{lmodern}

\usepackage{qcircuit}
\usepackage{braket}
\usepackage{mathdots}
\usepackage{comment}

%%%%%%%%%  Graphics path (to figure files)  %%%%%%%%%%%%%%%%%%%%%%%%%%%%%%%%

%% Can set graphicspath to point to specific directories containing figures (the current directory is searched automatically)
%% For instance, to search a subdirectory of the current directory called "figures" and a parallel directory called "art", set:

% \graphicspath{ {figures/} {../art/} }% For details see: https://latexref.xyz/dev/latex2e.html#g_t_005cgraphicspath


%%%%%%%%%  Representative set-up for biblatex  %%%%%%%%%%%%%%%%%%%%%%%%%%%%%

%% Numerical citations of references
\usepackage[style=ext-numeric-comp,giveninits=true,maxbibnames=10,sorting=none]{biblatex}

%% IEEE style citations and references
% \usepackage[style=ieee,maxbibnames=10,sorting=none]{biblatex}% style=ext-numeric-comp,articlein=false,giveninits=true
%	 \DefineBibliographyStrings{english}{url= \textsc{url} ,  }% replaces the IEEE default "[Online]. Available" by "URL"

%% author-year style citations and references 
%% use \parencite, not \cite, when you want "(Author, year)"
%% The sample files are not set up to include parentheses.
% \usepackage[style=authoryear, maxbibnames=10]{biblatex} 


\addbibresource{main.bib}%% <== change to YOUR bib file <= CHANGE

%% to avoid split urls and stretched white space, you can set the bibliography ragged-right:
%\appto{\bibsetup}{\raggedright}

%% biblatex is very powerful, and you can customize most aspects the reference list and citations to suit your needs.
%%   documentation is here: https://ctan.org/pkg/biblatex
%%   cheat sheet is here:   https://tug.ctan.org/info/biblatex-cheatsheet/biblatex-cheatsheet.pdf

%% To ensure citations are set, run Latex --> biblatex/biber --> Latex again

%%%%%%%%%%  Option to use natbib   %%%%%%%%%%%%%%%%%%%%%%%%%%%%%%%%%%%%%%%%%

%\RequirePackage[numbers,sort&compress]{natbib}
 
%%% add bibliography to table of contents
%\apptocmd{\bibliography}{\addcontentsline{toc}{chapter}{\protect\textbf{\bibname}}}{}{}

%%% You can use this to rename the bibliography section
%\renewcommand{\bibname}{References}

%%% To adjust space between bibliography items 
%\setlength\bibsep{4pt plus 1pt minus 1pt}
%   change 4pt to something else; don't drop last two lengths (they are stretchable "glue")


%%%%%%%%%%  Option for "double spacing" %%%%%%%%%%%%%%%%%%%%%%%%%%%%%%%%%%%%

%% Back in the typewriter era, double spaced lines were convenient for editing with a pencil. 
%% In typography, the separation between lines is called "leading", and it is usually set in 
%% proportion to the font size (i.e., when the font is loaded).  If you really feel the need 
%% to change the line separation, the most attractive results will be obtained by changing the
%% leading in proportion to the the current font size, rather than just doubling the space.

%% The setspace package provides a tool for changing line separation. Use these two commands here:
%
% \usepackage{setspace}%  documentation at https://ctan.org/pkg/setspace
% \setstretch{1.1}% you can choose some other value for the stretch of space between lines
%
%% Use one or more of the these commands *AFTER* the frontmatter
%
% \onehalfspacing
% \doublespacing
% \singlespacing  % will turn these effects off (you can use these anywhere in the document)

%% The best result is usually to stay with leading selected by the typographer who set up the font.


%%%%%%%%%%%  Metadata  %%%%%%%%%%%%%%%%%%%%%%%%%%%%%%%%%%%%%%%%%%%%%%%%%%%%%%%

% Most of the document metadata is created automatically. 
% The following items should be adjusted to match your work. <================= !!!!!!!!!!

\hypersetup{%
	pdfsubject={Optimizing Electronic-Structure Hamiltonians for Locality },
	% Change this to briefly state topic of your thesis 
% 
	pdfkeywords={Massachusetts Institute of Technology, MIT},
	% Add keywords that will help search engines and libraries to find your work.
	% Includes the name[s] of the author[s] 
	% (If you used \DocumentMetadata at line 14, you can just put "\CopyrightAuthor," for the names.)
%
	pdfurl={},
	% If you have a url for the thesis, put it here. Otherwise delete this.
	% (MIT Libraries will put your thesis in DSPACE with a persistent url after you submit it.)
%	
	pdfcontactemail={},
	% You can put a [permanent] email address into the metadata, if you like.
	% Otherwise delete this.
%
	pdfauthortitle={},
	% If you have a title, you can include it here.
}

%%%%%%%%%%%%  Math operators %%%%%%%%%%%%%%%%%%%%%%%%%%%%%%%%%%%%%%%%%%%%%%%%%%%%

% These commands declare two math operators, \erf{..} and \erfc{..} using mathtools
% note: * form produces automatic delimiter scaling; optional argument sets size manually, e.g. [\bigg]
% See Table 1.1 in Chapter 1



%%%%%%%%%%%%%%  End preamble %%%%%%%%%%%%%%%%%%%%%%%%%%%%%%%%%%%%%%%%%%%%%%%%%%%%%%%%%%%%%%%%%%%%%
%%%%%%%%%%%%%%%%%%%%%%%%%%%%%%%%%%%%%%%%%%%%%%%%%%%%%%%%%%%%%%%%%%%%%%%%%%%%%%%%%%%%%%%%%%%%%%%%%%

\begin{document}
%%% edit the following commands to match your thesis %%%%%%%%%%

\title{Optimizing Electronic-Structure Hamiltonians for Locality }

% \Author{Author full name}{Author department}[Author's first PREVIOUS degree][Author's second PREVIOUS degree][...
% Note that third, fourth, fifth, and sixth arguments are optional [] and may be omitted

% note on names: most of the following names are made up; Silas Holman was a physics professor at MIT in the 19th century.

\Author{Louis Wenjun Marquis}{Department of Electrical Engineering and Computer Science}[S.B. Computer Science and Engineering, Mathematics][Massachusetts Institute of Technology, 2025]
% \Author{Luisa Hernández}{Department of Research}[B.S. Mechanical Engineering, UCLA, 2018][M.S. Stellar Interiors, Vulcan Science Academy, 2020]
% \Author{Thurston Howell III}{Department of Economics}[MBA, Ferengi School of Management, 2022]

% Use once for each degree fulfilled by thesis
% For two degrees from one department, leave the department argument blank for the second degree {}.
\Degree{Master of Engineering in Electrical Engineering and Computer Science}{Department of Electrical Engineering and Computer Science}
%\Degree{Master of Science in Physics}{}
%\Degree{Bachelor of Science in Mechanical Engineering}{Department of Mechanical Engineering}

% If there is more than one supervisor, use the \Supervisor command for each.
\Supervisor{Aram Harrow}{Professor of Physics}
% \Supervisor{Edward C. Pickering}{Professor of Physics, and \\ \> Professor of Something Else}
% \Supervisor{Secunda Castor}{Professor of Research}
% \Supervisor{Quintus Castor}{Professor of Log Dams}

% Professor who formally accepts theses for your department (e.g., the Graduate Officer, Professor Sméagol,...)
% If more than one department, use more than once
\Acceptor{Katrina LaCurts}{Department of Electrical Engineering and Computer Science}{Chair, Master of Engineering Thesis Committee} % \\ \> Third title}
% \Acceptor{Quarta Castor}{Professor of Lodge Building}{Graduate Officer, Department of Mechanical Engineering}
%%%  If you need to reduce vertical space, put the acceptor title in the second argument and leave the third blank, {}.
% \Acceptor{Primus Castor}{Professor and Undergraduate Officer, Department of Physics}{}

% Usage: \DegreeDate{Month}{year}
% Valid degree months are February, May, June, or September
\DegreeDate{May}{2025}

% Date that final thesis is submitted to department
\ThesisDate{May 9, 2025}


%%%%%%  Choose whether to have a CREATIVE COMMONS License  %%%%%%%%%%%%%%%%%%%%%%%%%%%%%%%%%%%%%%
%
% If you are using a cc license, uncomment the following line and insert details of your cc license here.
%
% \CClicense{CC BY-NC-ND 4.0}{https://creativecommons.org/licenses/by-nc-nd/4.0/}
%

%%%%%%%  Solutions for overflowing titlepage  %%%%%%%%%%%%%%%%%%%%%%%%%%%%%%%%%%%%%%%%%%%%%%%%%%%

% If your title page is overflowing (from too many names, degrees, etc.):
%
% (a) you can reduce the 12pt and 18pt skips between various blocks to 6pt with this command:
%
% \Tighten
%
% (b)  you can scale down the Signature block at the bottom with this command:
%
% \SignatureBlockSize{\small}  %or this one \SignatureBlockSize{\footnotesize}
%
% (c) you can put the acceptor name and title onto two lines, rather than three like this:
%
% \Acceptor{Tertius Castor}{Professor and Graduate Officer, Department of Research}{}
%
% (d) you can change the font size of the author name[s] with
%
%	\AuthorNameSize{\normalsize}
%
% (e) and you can omit any previous degrees from the title page, instead mentioning them in the biographical sketch

% Also, if you prefer to keep the text toward the top of the page with most white space at the bottom, you
% can use this command to squash all of the vertical glue (stretchy space) with this command:
%
% \Squash 
%
% This command is useful when the text has not already reach the bottom of the page, since the glue gets squashed automatically
% when the page is too full.

%%%%%%%%%%%%%%%%%%%%%%%%%%%%%%%%%%%%%%%%%%%%%%%%%%%%%%%%%%%%%%%%%%%%%%%%%%%%%%%%%%%%%%%%%%%%%%%%%

%%% Make titlepage
\maketitle

%%%%%%%%% Contents that you need to write follows! %%%%%%%%%%%%%%%%%%%%%%%%%%%%%%%%%%%%%%%%%%%%%%

% \includeonly{acknowledgments,biography,chapter1,chapter2,...,appendixa,...} 
%   for usage of includeonly, see https://latexref.xyz/_005cinclude-_0026-_005cincludeonly.html

%%% Frontmatter (write this material in the mentioned files)  %%%%%%%%%%%%%%%%%%%%%%%%%%%%%%%%%%%

% This page is optional. Edit the file committee_members.tex 

% The abstract environment creates all the required headings and footers. 
% You only need to the text of the abstract in the file abstract.tex
\begin{abstract}
	% From mitthesis package
% Version: 1.01, 2023/06/19
% Documentation: https://ctan.org/pkg/mitthesis
%
% The abstract environment creates all the required headers and footnote. 
% You only need to add the text of the abstract itself.
%
% Approximately 500 words or less; try not to use formulas or special characters
% If you don't want an initial indentation, do \noindent at the start of the abstract

The developments of the ``kinetic theory'' of gases made within the last ten years have enabled it to account satisfactorily for many of the laws of gases. The mathematical deductions of Clausius, Maxwell and others, based upon the hypothesis of a gas composed of molecules acting upon each other at impact like perfectly elastic spheres, have furnished expressions for the laws of its elasticity, viscosity, conductivity for heat, diffusive power and other properties. For some of these laws we have experimental data of value in testing the validity of these deductions and assumptions. Next to the elasticity, perhaps the phenomena of the viscosity of gases are best adapted to investigation.\footnote{Text from Holman (1876): \doi{10.2307/25138434}.}  
% use \input rather than \include because we're inside an environment
\end{abstract}

%% acknowledgments.tex

% From mitthesis package
% Version: 1.02, 2024/06/19
% Documentation: https://ctan.org/pkg/mitthesis

\chapter*{Acknowledgments}
\pdfbookmark[0]{Acknowledgments}{acknowledgments}

I am grateful for Professor Aram Harrow who was both my instructor for the courses 8.370 and 8.372 at MIT and my thesis supervisor throughout my time as a Master of Engineering student. I sincerely appreciate the opportunity he provided to work as a research assistant in the Center for Theoretical Physics at the frontier of quantum computational chemistry, one of the most promising areas of quantum computing. Also crucial to this project was visiting PhD student Jakob Guenther from the University of Copenhagen. I appreciate his generosity with his time to have recurring meetings to transmit to me his expertise on this specific problem both in person and across a six-hour time difference. I also congratulate him on his upcoming post-doctorate position at MIT. I want to thank both the Center for Theoretical Physics at MIT and the Quantum for Bio project at the University of Copenhagen, as each provided half of the funding for this project.
I also want to thank Professor Marek Perkowski of Portland State University, who taught me quantum computing before I came to MIT. Without his instruction and the work that we did together, I would not have arrived at this point in my academic career.% acknowledgments.tex (.tex extension is presumed by \include) 

%%% Table of contents and lists of stuff (delete unused lists, i.e., if no tables or figures) %%%%%

\tableofcontents
\listoffigures
\listoftables

%%% Chapters of thesis  %%%%%%%%%%%%%%%%%%%%%%%%%%%%%%%%%%%%%%%%%%%%%%%%%%%%%%%%%%%%%%%%%%%%%%%%%%%

%% If you want to use "double spacing", you should start here...

% From mitthesis package
% Version: 1.07, 2024/09/26
% Documentation: https://ctan.org/pkg/mitthesis


\chapter{Introduction}

The field of computational chemistry aims to use computers to simulate and better understand the behavior of complex chemical systems in the real world. Drug discovery, materials science, and nanotechnology are only a few areas that would benefit from the ability to computationally predict behavior of molecular systems. However, chemical systems at the particle level are inherently shaped by quantum effects, which are not easily modeled by a classical (non-quantum) computer. For example, the phenomenon of quantum superposition allows a system to be in a linear combination of an exponential number of basis states at the same time. A classical computer must numerically store this exponential amount of data, a prohibitive task.

Quantum computational chemistry aims to perform this task using quantum computers, which can represent such systems more easily because its hardware, by definition, is inherently quantum. A superposition in a chemical system, for example, can represented by physically putting a quantum computer's hardware into superposition.

\section{Hamiltonian Simulation}

In any quantum system, the behavior in which its state changes over time is dictated by its Hamiltonian operator $H$. In particular, if the system starts in state $\ket{\psi(0)}$, then we can determine its state $\ket{\psi(t)}$ at time $t$.

\begin{equation}
    \ket{\psi(t)} = e^{-iHt}\ket{\psi(0)}
\end{equation}

A wavefunction maps a continuous physical quantity (such as position) to an amplitude. A state denotes a physical system's wavefunction at a particular moment. An operator linearly maps wavefunctions to wavefunctions like how a matrix maps vectors to vectors.

The Hamiltonian $H$ depends on many factors, such as the nature of interactions between particles in the system or external forces such as electric or magnetic fields acting in the system. $H$, and in turn $U = e^{-iHt}$ can be very complex, and usually it's not feasible to simulate $U$ exactly. Instead, we try to find a different operator $\tilde{U}$ that is easy to simulate and sufficiently close to $\tilde{U}$. Usually we choose to define the error as the spectral norm (largest singular value) of the difference between $U$ and $\tilde{U}$. The goal is for the error to be below some variable threshold $\epsilon$.

\begin{equation}
    \epsilon \geq ||U - \tilde{U}|| = ||e^{-iHt} - \tilde{U}||
\end{equation}

This task is accordingly named Hamiltonian simulation.

One important Hamiltonian is the electronic structure Hamiltonian, which describes the behavior of electrons in a molecule consisting of many nuclei and electrons. In modeling this system, we use the Born-Oppenheimer approximation, which assumes that the positions of the nuclei are essentially fixed. This is justified by the fact that the nuclei are far more massive than the electrons, so we can study the movement of electrons separately from the movement of the nuclei. With this assumption, the electronic structure Hamiltonian contains only terms for the electrons' kinetic energies, for the interactions between electrons and nuclei, and for the interactions among electrons.

\begin{equation}
    H = -\sum_p \frac{\hbar^2}{2m_e} \nabla^2_p - \sum_{p, P}\frac{e^2}{4\pi\epsilon_0}\frac{Z_P}{|\vec{r}_p - \vec{R}_P|} + \frac{1}{2}\sum_{p \neq q}\frac{e^2}{4\pi\epsilon_0}\frac{1}{|\vec{r}_p - \vec{r}_q|} \label{eq: H}
\end{equation}

$\vec{r}_p$ denotes the position of the $p$th electron, and $\vec{r}_P$ and $Z_P$ denote the position and charge of the $P$th nucleus.

The ultimate goal of electronic structure theory is to find the eigenvalues and eigenstates of $e^{-iHt}$, or the states that are invariant (up to a phase) under $e^{-iHt}$. The ability to simulate the behavior of $e^{-iHt}$ on an arbitrary state can directly be used to find the eigenvalues using the quantum phase estimation algorithm. The details of this algorithm are explained in \cite{MikeIke}; the relevant fact for this paper is the power unlocked by performing $e^{-iHt}$.

\section{Second Quantized Fock States}

We would like to model the electronic structure system described by $H$ in \eqref{eq: H}. As previously described, the system consists of numerous identical electrons and numerous nuclei at fixed positions. The indistinguishability of the electrons means that the joint wavefunction is only determined by how many electrons are in each basis state, rather than which electron is in which state. For example, electrons $1$ and $2$ being in respective states $\phi_1$ and $\phi_2$ must create the same wavefunction as them being in the states $\phi_2$ and $\phi_1$. In addition, the joint wavefunction of many identical fermions must be anti-symmetric upon exchange of any pair of fermions. For example, if we want to express a state where one electron is in state $\phi_1$ and the other is in state $\phi_2$, a simple tensor product $\phi_1 \otimes \phi_2$ would not be a valid joint wavefunction. Instead, it must be anti-symmetrized.

\begin{equation}
    \psi = \phi_1 \otimes \phi_2 - \phi_2 \otimes \phi_1
\end{equation}

This required process can make many-body calculations very cumbersome. Fortunately, second quantization significantly simplifies the representation of many-body systems by introducing the Fock basis, whose states are inherently anti-symmetrized.

Suppose for a single electron we'd like to use a basis $\{\phi_i\}$. Although the basis has infinite size, usually only basis states below a certain energy threshold are relevant, restricting our concern to a finite basis $\{\phi_i\}_{i \in [0, n)}$.

If there are many identical particles, we often want to express a state where $f_i$ particles are in the $\phi_i$ state. We can label such a state with a vector $\vec{f}$. This is a Fock state $\ket{\vec{f}}$. For fermions (such as electrons), $f_i \in \{0, 1\}$ due to the Pauli-Exclusion Principle, so $\vec{f} \in \{0, 1\}^n$.

\begin{equation}
    \ket{\vec{f}} = \ket{f_0, f_1, ..., f_{n - 1}} \label{eq: fock}
\end{equation}

With $n$ orbitals (basis states), there are $2^n$ such Fock states, one for each n-bitstring $\vec{f} \in \{0, 1\}^n$. These $2^n$ states form a complete basis of the many-body Hilbert space. That is, any state possible for a many-body system within these $n$ orbitals can be expressed as a linear combination of these Fock states.

We can define fermionic creation $a^\dag_i$ and annihilation $a_i$ operators that map Fock states to Fock states.

\begin{equation}
    \begin{split}
        a^\dag_i\ket{\vec{f}} &= a^\dag_i\ket{f_0, ..., f_{n - 1}} \\
        &= \delta_{f_i, 1}(-1)^{\sum_{j = 0}^{i - 1}}\ket{f_0, ..., f_i \oplus 1, ..., f_{n - 1}}
    \end{split}
    \label{eq: creation}
\end{equation}
\begin{equation}
    \begin{split}
        a_i\ket{\vec{f}} &= a_i\ket{f_0, ..., f_{n - 1}} \\
        &= \delta_{f_i, 0}(-1)^{\sum_{j = 0}^{i - 1}}\ket{f_0, ..., f_i \oplus 1, ..., f_{n - 1}}
    \end{split}
    \label{eq: annihilation}
\end{equation}

$a^\dag_i$ transforms a Fock state with $f_i = 0$ into a Fock state with $f_i = 1$, thereby ``creating'' a fermion in state $\phi_i$ and adding a phase shift to account for anti-symmetry. If orbital $i$ is already occupied, the new state is $0$ (it disappears). $a_i$ does the reverse of $a^\dag_i$ and de-occupies orbital $i$.

As typical with creation and annihilation operators, their product is a number operator $n_i = a^\dag_ia_i$ \cite{QCC}.

\begin{equation}
    n_i\ket{\vec{f}} = f_i\ket{f_0, ..., f_{n - 1}}
\end{equation}

The creation and annihilation operators have anti-commutation relations \cite{QCC} that are analogous to typical commutation relations of creation and annihilation operators.

\begin{equation}
    \{a^\dag_i, a_j\} = \delta_{i, j}
\end{equation}
\begin{equation}
    \{a_i, a_j\} = \{a^\dag_i, a^\dag_j\} = 0
\end{equation}

Second quantization, in short, equips us with a concise representation of many-fermion wavefunction in the form of Fock states and operators $a^\dag_i, a_i$ to easily manipulate them.

\section{Jordan-Wigner Transformation}

The Fock basis allows us to easily represent physical systems of many electrons. We would like to encode these physical systems on a quantum computer made of qubits so that we can compute on them. Conveniently, it is very straightforward to do so with the Fock basis.

Quantum computers are comprised of qubits. A qubit is simply a system with a two-dimensional Hilbert space. One example is an electron spin (up or down). Given a basis $\{\ket{0}, \ket{1}\}$, the state of a qubit can be any normalized linear combination of these two basis vectors.

\begin{equation}
    \ket{\psi} = A_0\ket{0} + A_1\ket{1} \text{ such that } |A_0|^2 + |A_1|^2 = 1
\end{equation}

The joint state of $n$ qubits can be any normalized superposition of all $2^n$ basis states $\{\ket{\vec{x}}\}_{\vec{x} \in \{0, 1\}^n} = \{\ket{x_0, x_1, ..., x_{n - 1}}\}_{\vec{x} \in \{0, 1\}^n}$.

\begin{equation}
    \ket{\psi} = \sum_{\vec{x} \in \{0, 1\}^n} A_{\vec{x}} \ket{\vec{x}} \text{ such that } \sum_{\vec{x} \in \{0, 1\}^n} |A_{\vec{x}}|^2 = 1 \label{eq: qubits}
\end{equation}

It is then straightforward that basis states \eqref{eq: fock} of an $n$-orbital many-electron physical system correspond to basis states \eqref{eq: qubits} of $n$ qubits in a quantum computer. This is the Jordan-Wigner Encoding.

\begin{equation}
    \ket{\vec{f}} \leftrightarrow \ket{\vec{x}}
\end{equation}

That is, in the Jordan-Wigner Encoding, the value $x_i$ of the $i$th qubit represents the occupation $f_i$ of the orbital $\phi_i$.

Accordingly, creation \eqref{eq: creation} and annihilation \eqref{eq: annihilation} operators on the physical system also correspond to qubit operators.

\begin{equation}
    a_i \leftrightarrow Z_0Z_1...Z_{i - 1}\ket{0}\bra{1}_i \label{eq: JWdown}
\end{equation}
\begin{equation}
    a^\dag_i \leftrightarrow Z_0Z_1...Z_{i - 1}\ket{1}\bra{0}_i \label{eq: JWup}
\end{equation}
\begin{equation}
    n_i = a^\dag_ia_i \leftrightarrow \ket{1}\bra{1}_i \label{eq: JWnum}
\end{equation}

We use the short-hand $U_i = I^{\otimes i}\otimes U \otimes I^{\otimes n - i - 1}$ to denote applying a gate $U$ on the $i$th qubit.

Observe that individual physical and annihilation operators map to a series of $i$ gates (qubit operators). In the average case, this amounts to $O(n)$ gates, which is not ideal. There are other encodings that optimize the efficiency of operator mappings, but they sacrifice the simplicity in basis state mappings exhibited by the Jordan-Wigner Encoding. For the purposes of this project, the Jordan-Wigner Encoding is sufficient.

\section{Second-Quantized Electronic Structure Hamiltonian}

Equipped with Fock states and creation and annihilation operators, we can rewrite the electron structure Hamiltonian from \eqref{eq: H} in a simpler, second-quantized form.

\begin{equation}
    H = \sum_{i, j} h_{ij}a^\dag_ia_j + \frac{1}{2}\sum_{i,j,k,l} h_{ijkl}a^\dag_ia^\dag_ja_ka_l
\end{equation}

The real coefficients $h_{ij}$ and $h_{ijkl}$ are the projections of $H$ onto the chosen single-particle basis $\{\phi_i\}_{i \in [0, n)}$. These can be calculated on a classical computer.

\begin{equation}
    \begin{split}
        h_{ij} &= \braket{\phi_i | \left(-\frac{\hbar^2}{2m_e}\nabla^2 - \sum_I \frac{e^2}{4\pi\epsilon_0}\frac{Z_I}{|\vec{r} - \vec{R}_I|}\right) | \phi_j} \\
        &= \int \phi^*_i(\vec{r})\left(-\frac{\hbar^2}{2m_e}\nabla^2 - \sum_I \frac{e^2}{4\pi\epsilon_0}\frac{Z_I}{|\vec{r} - \vec{R}_I|}\right)\phi_j(\vec{r}) d^3\vec{r}
    \end{split}
    \label{eq: hij}
\end{equation}
\begin{equation}
    \begin{split}
        h_{ijkl} &= \braket{\phi_i, \phi_j | \frac{e^2}{4\pi\epsilon_0}\frac{1}{|\vec{r}_1 - \vec{r}_2|} | \phi_l, \phi_k} \\
        &= \iint \phi^*_i(\vec{r}_1)\phi^*_j(\vec{r}_2)\frac{e^2}{4\pi\epsilon_0}\frac{1}{|\vec{r}_1 - \vec{r}_2|}\phi_l(\vec{r}_1)\phi_k(\vec{r}_2) d^3\vec{r}_1d^3\vec{r}_2
    \end{split}
    \label{eq: hijkl}
\end{equation}

Observe from \eqref{eq: hij} that $h_{ij}$, which is real, must be symmetric between $i$ and $j$. From \eqref{eq: hijkl}, $h_{ijkl}$, which is also real, must also be symmetric between $i$ and $l$, $j$ and $k$, and $(i, l)$ and $(j, k)$. Inspired by this symmetry, we can rearrange the operators in the second summation using anti-commutation relations to put $i, l$ and $j, k$ next to each other.

\begin{equation}
    \begin{split}
        H &= \sum_{i, j} h_{ij}a^\dag_ia_j + \frac{1}{2}\sum_{i,j,k,l} h_{ijkl}a^\dag_ia^\dag_ja_ka_l \\
        &= \sum_{i, j} h_{ij}a^\dag_ia_j - \frac{1}{2}\sum_{i,j,k,l} h_{ijkl}a^\dag_ia^\dag_ja_la_k \\
        &= \sum_{i, j} h_{ij}a^\dag_ia_j - \frac{1}{2}\sum_{i,j,k,l} h_{ijkl}a^\dag_i(\delta_{jl} - a_la^\dag_j)a_k \\
        &= \sum_{i, j} h_{ij}a^\dag_ia_j - \frac{1}{2}\left(\sum_{i, j, k}h_{ijkj}a^\dag_ia_k - \sum_{i,j,k,l} h_{ijkl}a^\dag_ia_la^\dag_ja_k\right) \\
        &= \sum_{i, j} (h_{ij} - \frac{1}{2}\sum_k h_{ikjk})a^\dag_ia_j + \frac{1}{2}\sum_{i,j,k,l} h_{ijkl}a^\dag_ia_la^\dag_ja_k \\
        &= \sum_{i, j} h'_{ij}a^\dag_ia_j + \frac{1}{2}\sum_{i,j,k,l} h'_{iljk}a^\dag_ia_la^\dag_ja_k \\
        &= H_{1e} + H_{2e}
    \end{split}
    \label{eq: H3}
\end{equation}
We introduced new coefficients $h'_{ij}$ and $h'_{iljk}$ for this new ordering and label the summations $H_{1e}$ and $H_{2e}$.
\begin{equation}
    \begin{split}
        h'_{ij} &= h_{ij} - \frac{1}{2}\sum_k h_{ikjk} \\
        H_{1e} &= \sum_{i, j} h'_{ij}a^\dag_ia_j \label{eq: H1}
    \end{split}
\end{equation}
\begin{equation}
    \begin{split}
        h'_{iljk} &= h_{ijkl} \\
        H_{2e} &= \frac{1}{2}\sum_{i,j,k,l} h'_{iljk}a^\dag_ia_la^\dag_ja_k \label{eq: H2}
    \end{split}
\end{equation}

\section{Trotterization}

We would like to simulate $U = e^{-iHt}$ on a quantum computer. Unfortunately, simulating $U$ itself is very difficult: we don't even know what the eigenstates of $H$ are. The main problem is $H$ is sum of many terms, each of which act very differently on a state. We'd like to study each term separately. Fortunately, Trotterization allows us to do so.

Consider a generic Hamiltonian $H$ that is the sum of $L$ terms $h_j$.

\begin{equation}
    H = \sum_{j = 0}^{L - 1} h_j
\end{equation}

We can use the Lie-Trotter formula to approximate the target operator $U = e^{-iHt}$ as the product of many smaller rotations.

\begin{equation}
    U = e^{-iHt} = e^{-i\sum_{j = 0}^{L - 1} h_jt} = \lim_{M \rightarrow \infty} (\prod_{j = 0}^{L - 1} e^{-ih_j\frac{t}{M}})^M
\end{equation}

In other words, for sufficiently large $M$, $U$ is essentially equivalent to iterating $M$ times through smaller evolutions $e^{-ih_j\frac{t}{M}}$. We can denote $\Delta t = \frac{t}{M}$ the step size. This is the first-order Trotter-Suzuki Algorithm. One can use the term ``Trotterize'' to denote the breaking of $e^{-iHt}$ into the product of many $e^{-ih_j\Delta t}$

However, $M$ must be sufficiently large to make the approximation sufficiently precise. In first-order Trotterization, $M = O(\frac{(L\Lambda t)^2}{\epsilon})$ (in which $\lambda = max_j ||h_j||$, the largest singular value of a term), scales quadratically with the number of terms $L$. Each iteration requires $L$ steps, totalling $O(ML = O(\frac{L^3(\Lambda t)^2}{\epsilon}))$ steps.

We can reduce the asymptotic number of iterations necessary with the second-order Trotter-Suzuki Algorithm, which essentially iterates $[h_j]_j$ in ascending order then descending order.

\begin{equation}
    U = e^{-iHt} = e^{-i\sum_{j = 0}^{L - 1} h_jt} = \lim_{M \rightarrow \infty} (\prod_{j = 0}^{L - 1} e^{-ih_j\frac{t}{2M}}\prod_{j = L - 1}^{0} e^{-ih_j\frac{t}{2M}})^M
\end{equation}

Second-order Trotterization requires $M = O(\frac{(L\Lambda t)^{1.5}}{\sqrt{\epsilon}})$. In fact, one can continue increasing the order: in general, the $2k$th order Trotterization requires $M = O(\frac{(L\Lambda t)^{1+\frac{1}{2k}}}{\epsilon^{\frac{1}{2k}}})$. However, increasing the order $k$ quickly increases the constant factor, so in practice it's rarely useful to go beyond $k = 3$.

Randomizing the order of the steps $e^{-ih_j\Delta t}$ can significantly improve performance. \cite{QDRIFT} proposed the qDRIFT algorithm, which samples among terms $[h_j]_j$, assigning $h_j$ a probability proportional to $||h_j||$. It requires at most $\frac{2\lambda^2 t^2}{\epsilon} \leq \frac{2L^2\Lambda^2 t^2}{\epsilon}$ total steps, in which $\lambda = \sum_j ||h_j||$.

The power of such techniques like Trotterization and qDRIFT (breaking $e^{-iHt}$ into many $e^{-ih_j\Delta t}$) allows us to analyze each term $h_j$ independently of the others. If one can simulate $e^{-ih_j \Delta t}$ for arbitrary $\Delta t$ for every term, one can approximately simulate $U = e^{-iHt}$.

For example, one can Trotterize the electronic structure Hamiltonian \eqref{eq: H3}. One can substitute the operator correspondances from \eqref{eq: JWdown} and \eqref{eq: JWup} into \eqref{eq: H3} to obtain the corresponding qubit Hamiltonian, which would be a sum of many tensor products of Paulis. However, with $O(n^4)$ terms in \eqref{eq: H2}, there end up being $L = O(n^4)$ tensor products of Paulis in the qubit Hamiltonian, each containing $O(n)$ non-identity Paulis. It is possible to reduce this $O(n)$ to $O(1)$ amortized through methods in \cite{Amortize1}, \cite{Amortize2}, \cite{Amortize3}, or with a an FSN-method adapted from \cite{FSN}. Regardless, with $L = O(n^4)$, the number of iterations with second-order Trotter-Suzuki becomes $M = O(L^{1.5}) = O(n^6)$, totalling $O(ML) = O(n^{10})$ total steps. In the rest of this paper, we will refer to this direct Trotterization as the ``brute force method'' We'd like to see if there's a more concise method.

\section{Double Factorization of Electronic Structure Hamiltonian}

Recall that $h'_{iljk} = h_{ijkl}$ is symmetric between $(i, l)$ and $(j, k)$. $h'$ can then be treated as a real symmetric $n^2 \times n^2$ matrix and eigendecomposed with eigenvalues $\lambda_r$ and eigenvectors $Q^{(r)}$ whose indices are $(i, l)$ ordered pairs. A Cholesky decomposition is also an option but for now we will use an eigendecomposition. Denote $R = \text{rank}{(h')}$.

\begin{equation}
    h'_{iljk} = \sum_{r = 0}^{R - 1} \lambda_r Q^{(r)}_{i, l}Q^{(r)}_{j, k}
\end{equation}

\begin{equation}
    \begin{split}
        H_{2e} &= \frac{1}{2}\sum_{i,j,k,l} h'_{ijkl}a^\dag_ia_la^\dag_ja_k \\
        &= \frac{1}{2}\sum_{i,j,k,l} \sum_{r = 0}^{R - 1} \lambda_r Q^{(r)}_{i, l}Q^{(r)}_{j, k}a^\dag_ia_la^\dag_ja_k \\
        &= \frac{1}{2}\sum_{r = 0}^{R - 1} \lambda_r \sum_{i,l} Q^{(r)}_{i, l}a^\dag_ia_l\sum_{j,k}Q^{(r)}_{j, k}a^\dag_ja_k \\
        &= \frac{1}{2}\sum_{r = 0}^{R - 1}\lambda_r\left(\sum_{i,j} Q^{(r)}_{i, j}a^\dag_ia_j\right)^2 \\
        &= \sum_{r = 0}^{R - 1} H_{2e}^{(r)}
    \end{split}
    \label{eq: DF}
\end{equation}

We have written $H_{2e}$ as a sum of terms $H_{2e}^{(r)}$, defined in \eqref{eq: H2er}. The number $R$ of terms is the rank of $h'$, and in the worst case $R = n^2$. However, for molecular systems, $h'$ is often low-rank, with $R$ often being $O(n)$, such as in the carbon-hydrogen systems studied in \cite{CH}. If we could simulate $U_2^{(r)} = e^{-iH_{2e}^{(r)}\Delta t}$ for an arbitrary $\Delta t$, then we could Trotter or qDRIFT over these $R$ terms to implement $U_2$, far fewer than the $O(n^4)$ terms in direct Trotterization of $H_{2e}$.

\begin{equation}
    H_{2e}^{(r)} = \frac{1}{2}\lambda_r\left(\sum_{i,j} Q^{(r)}_{i, j}a^\dag_ia_j\right)^2 \label{eq: H2er}
\end{equation}

Observe that $h'_{iljk}$ is symmetric between $i$ and $l$ and between $j$ and $k$, implying that $Q^{(r)}_{i, j}$ is symmetric between $i$ and $j$. Each $n^2$-length vector $Q^{(r)}$ can then be treated as a symmetric $n \times n$ matrix to be further eigendecomposed with eigenvalues $\lambda'^{(r)}_s$ and real orthonormal eigenvectors $u^{(r)}_s$ as the columns of $u^{(r)}$, which is orthogonal.

\begin{equation}
    \begin{split}
        Q^{(r)} &= u^{(r)}\begin{bmatrix}
            \lambda'^{(r)}_0 & & \\ & \ddots & \\ & & \lambda'^{(r)}_{n - 1}
          \end{bmatrix}u^{(r)T} \\
        Q^{(r)}_{i, j} &= \sum_s \lambda'^{(r)}_s u^{(r)}_{is}u^{(r)}_{js}
    \end{split}
\end{equation}

We introduce the operators $\tilde{a}^{(r)\dag}_s, \tilde{a}^{(r)}_s$, which are respectively linear combinations of the the creation operators $\{a^\dag_i\}_{i \in [0, n)}$ and annihilation operators $\{a_i\}_{i \in [0, n)}$. Because $u^{(r)}$ is orthogonal (and therefore unitary), these $\tilde{a}^{(r)\dag}_s, \tilde{a}^{(r)}_s$ are actually creation and annihilation operators of a different Fock basis; that is, the Fock basis created from the single-particle basis $\{\tilde{\phi}_s = \sum_s u^{(r)}_{is} \phi_i\}_{s \in [0, n)}$. This rotated basis also has its own number operators $\tilde{n}^{(r)}_s$.

\begin{equation}
    \begin{split}
        \tilde{a}^{(r)\dag}_s &= \sum_iu^{(r)}_{is}a^\dag_i \\
        \tilde{a}^{(r)}_s &= \sum_iu^{(r)}_{is}a_i \\
        \tilde{n}^{(r)}_s &= \tilde{a}^{(r)\dag}_s\tilde{a}^{(r)}_s
    \end{split}
\end{equation}

We can rewrite $H_{2e}^{(r)}$ in terms of $\tilde{a}^{(r)\dag}_s, \tilde{a}^{(r)}_s, \tilde{n}^{(r)}_s$.

\begin{equation}
    \begin{split}
        H_{2e}^{(r)} &= \frac{1}{2}\lambda_r\left(\sum_{i,j} Q^{(r)}_{i, j}a^\dag_ia_j\right)^2 \\
        &= \frac{1}{2}\lambda_r\left(\sum_{i,j} \sum_s \lambda'^{(r)}_s u^{(r)}_{is}u^{(r)}_{js}a^\dag_ia_j\right)^2 \\
        &= \frac{1}{2}\lambda_r\left(\sum_s \lambda'^{(r)}_s \sum_i u^{(r)}_{is}a^\dag_i\sum_ju^{(r)}_{js}a_j\right)^2 \\
        &= \frac{1}{2}\lambda_r\left(\sum_s \lambda'^{(r)}_s \tilde{a}^{(r)\dag}_s\tilde{a}^{(r)}_s\right)^2 \\
        &= \frac{1}{2}\lambda_r\left(\sum_s \lambda'^{(r)}_s \tilde{n}^{(r)}_s\right)^2
    \end{split}
\end{equation}

We would like to simulate $U_2^{(r)} = e^{-iH_{2e}^{(r)}\Delta t}$ for an arbitrary $\Delta t$. If the operator's eigenbasis were the same as the qubits' computational basis, it would be very simple. However, its eigenbasis is the Fock basis constructed from the rotated basis $\{\tilde{\phi}_s\}$. Meanwhile, the qubit computational basis $\{\ket{\vec{x}}\}$ corresponds to the original physical Fock basis $\{\ket{\vec{f}}\}$ constructed from $\{\phi_i\}$. To deal with this, we have to essentially rotate the operator $U_r$ into the computational basis. \cite{FSN} uses the Thouless Theorem to show that such a rotation operator $U_R(u^{(r)})$ necessary can be calculated from $u^{(r)}$ and and corresponds to a maximum of ${n \choose 2} = O(n^2)$ gates on the quantum computer. 

\begin{equation}
    U_2^{(r)} = e^{-\frac{i\Delta t}{2}\lambda_r\left(\sum_s \lambda'^{(r)}_s \tilde{n}^{(r)}_s\right)^2} = U_R(u^{(r)})^\dag e^{-\frac{i\Delta t}{2}\lambda_r\left(\sum_s \lambda'^{(r)}_s n_s\right)^2}U_R(u^{(r)}) = U_R(u^{(r)})^\dag U_A^{(r)}U_R(u^{(r)})
    \label{eq: U_2^r}
\end{equation}
\begin{equation}
    U_A^{(r)} = e^{-\frac{i\Delta t}{2}\lambda_r\left(\sum_s \lambda'^{(r)}_s n_s\right)^2}
\end{equation}

\subsection{Double-Factorized Expansion Method}

We'd like to find a qubit analog to $U_A^{(r)}$. One way is to expand the square inside $U_A^{(r)}$ and factor it into ${n + 1 \choose 2}$ individual operators, since the terms all commute. Let's call this the ``double-factorized expansion method'', or ``expansion method'' for short.

\begin{equation}
    \begin{split}
        U_A^{(r)} &= e^{-\frac{i\Delta t}{2}\lambda_r\left(\sum_s \lambda'^{(r)}_s n_s\right)^2} \\
        &= e^{-\frac{i\Delta t}{2}\lambda_r\sum_{s, s'} \lambda'^{(r)}_s\lambda'^{(r)}_{s'} n_sn_{s'}} \\
        &= \prod_{s, s'} e^{-\frac{i\Delta t}{2}\lambda_r\lambda'^{(r)}_s\lambda'^{(r)}_{s'} n_sn_{s'}} \\
        &= \prod_s e^{-\frac{i\Delta t}{2}\lambda_s(\lambda'^{(r)}_s)^2 n_s}\prod_{s < s'} e^{-i\Delta t\lambda_r\lambda'^{(r)}_s\lambda'^{(r)}_{s'} n_sn_{s'}}
        \label{eq: expansion}
    \end{split}
\end{equation}

Each operator with a single $n_s$ corresponds (up to a phase) to a $R_z$ operator on the $s$th qubit in a quantum computer. Each operator with $n_sn_{s'}$ corresponds (up to a phase) to a controlled $R_z$ operator on the $s$th and $s'$th qubits in a quantum computer. 
\begin{equation}
    e^{-\frac{i\Delta t}{2}\lambda_r(\lambda'^{(r)}_s)^2 n_s} \leftrightarrow R_z(-\frac{\Delta t}{2}\lambda_r(\lambda'^{(r)}_s)^2)_s
\end{equation}
\begin{equation}
    R_z(\varphi) = \begin{bmatrix}
        e^{-i\frac{\varphi}{2}} & 0 \\ 0 & e^{i\frac{\varphi}{2}}
    \end{bmatrix} = e^{-i\frac{\varphi}{2}}\begin{bmatrix}
        1 & 0 \\ 0 & e^{i\varphi}
    \end{bmatrix} = e^{-i\frac{\varphi}{2}}(\ket{0}\bra{0} + e^{i\varphi}\ket{1}\bra{1})
\end{equation}
\begin{equation}
    e^{-\frac{i\Delta t}{2}\lambda_r\lambda'^{(r)}_s\lambda'^{(r)}_{s'} n_sn_{s'}} \leftrightarrow CR_z(-\frac{\Delta t}{2}\lambda_r\lambda'^{(r)}_s\lambda'^{(r)}_{s'})_{s, s'}
\end{equation}
\begin{equation}
    CR_z(\varphi) = \begin{bmatrix}
        1 & 0 & 0 & 0 \\ 0 & 1 & 0 & 0 \\ 0 & 0 & 1 & 0 \\ 0 & 0 & 0 & e^{i\varphi}
    \end{bmatrix}
    \label{eq: CR_z}
\end{equation}

There are therefore $n$ $R_z$ operators and ${n \choose 2}$ $CR_z$ operators in this decomposition. Each $CR_z$ can be decomposed (up to a phase) to two $CNOT$s and three $R_z$s, shown in \ref{fig: CR_z}. This results in $n + 3{n \choose 2} = \frac{3n(n - 1)}{2}$ $R_z$ gates total in $U_A^{(r)}$.

\begin{figure}[h]
    \centering
    \Qcircuit @C=1em @R=1.6em {
        \lstick{} & \gate{R_z(\frac{\varphi}{4})} & \ctrl{1} & \qw & \ctrl{1} & \qw \\
        \lstick{} & \gate{R_z(\frac{\varphi}{4})} & \targ    & \gate{R_z(-\frac{\varphi}{4})} & \targ & \qw 
    }
    \caption{Construction of $CR_z(\varphi)$ in \eqref{eq: CR_z} with two $CNOT$ gates and three $R_z$ gates}
    \label{fig: CR_z}
\end{figure}

\cite{Rz} achieves a fault-tolerant implementation of $R_z$ using at most $10 + 4\log_2(\frac{1}{\epsilon_z})$ T-gates, in which $\epsilon_z$ is the the desired error bound of $R_z$. Since $U_A^{(r)}$ contains $\frac{3n(n - 1)}{2}$ $R_z$ gates and error (which is by operator norm) propagates sub-additively, we need $\epsilon_z \leq \frac{\epsilon}{\frac{3n(n - 1)}{2}}$ to guarantee a threshold $\epsilon$ for the error of $U_A^{(r)}$.

We can then calculate the upper bound for the $T$-count $\tilde{N}_T$ of $U_A^{(r)}$ using the expansion method in terms of $n, \frac{1}{\epsilon}$.

\begin{equation}
    \begin{split}
        \tilde{N}_T &= \frac{3n(n - 1)}{2}(10 + \log_2(\frac{1}{\epsilon_z})) \\
        &= \frac{3n(n - 1)}{2}(10 + \log_2(\frac{\frac{3n(n - 1)}{2}}{\epsilon})) \\
        &= O(n^2\log_2(\frac{n}{\epsilon}))
    \end{split}
    \label{eq: expansionTs}
\end{equation}

\section{Discussion}

Similarly to \eqref{eq: expansion}, $U_R(u^{(r)})$ in \eqref{eq: U_2^r} can be decomposed into at most $O(n^2)$ non-Clifford gates as described in \cite{FSN}, resulting in a $O(n^2\log_2(\frac{n}{\epsilon}))$ $T$-count. Combined with \eqref{eq: expansionTs}, one can achieve an overall $T$-count of $O(n^2\log_2(\frac{n}{\epsilon}))$ for $U_2^{(r)}$.

We'd like to find alternative fault-tolerant implementations of $U_2^{(r)}$ with a lower asymptotic $T$-count. For this to be possible, the chemical system must have a Hamiltonian with ``easy'' basis rotations; that is, it double factorizes such that all $U_R(u^{(r)})$ decomposes using the \cite{FSN} method to $o(n^2)$ non-Clifford gates. We also need to both implement $U_A^{(r)}$ more efficiently.

Chapter 2 introduces and explicitly constructs a quantum circuit that simulates $U_R(u^{(r)})$ with $O(n\log_2(\frac{n}{\epsilon}) + \log_2(\frac{1}{\epsilon})^2)$ $T$-count. We call this quantum circuit the ``coherent method'', in contrast to the ``expansion method''. Chapter 3 explores types of Hamiltonians with ``easy'' basis rotations and how to effectively leverage the ``coherent method'' for these Hamiltonians.

%%%%%%%%%%%%%%%% end table %%%%%%%%%%%%%%%%%%% 

% .tex extension is presumed
% From mitthesis package
% Version: 1.07, 2024/09/26
% Documentation: https://ctan.org/pkg/mitthesis


\chapter{Quantum Arithmetic Circuit Design for Double-Factorized Electronic Structure Hamiltonian Simulation}

We now propose and analyze a quantum circuit that approximately simulates $U_A^{(r)}$ from \eqref{eq: brute}.

\begin{equation}
    U_A^{(r)} = e^{-\frac{i\Delta t}{2}\lambda_r\left(\sum_s \lambda'^{(r)}_s n_s\right)^2} \label{eq: U_A}
\end{equation}

Observe that because $U_A^{(r)}$ is comprised of number operators $n_s$, its eigenbasis is simply the Fock basis $\{\ket{\vec{f}}\}_{\vec{f}}$. So when it acts on a Fock state $\ket{\vec{f}}$, its effect is simply a phase shift depending on the bits of $\vec{f}$.

\begin{equation}
    U_A^{(r)}\ket{\vec{f}} = e^{-\frac{i\Delta t}{2}\lambda_r\left(\sum_s \lambda'^{(r)}_s n_s\right)^2}\ket{\vec{f}} = e^{-\frac{i\Delta t}{2}\lambda_r\left(\sum_s \lambda'^{(r)}_s f_s\right)^2}\ket{\vec{f}}
\end{equation}

We can substitute the number operator correspondance from \eqref{eq: JWnum} to find the equivalent qubit operator.

\begin{equation}
    U_A^{(r)} = e^{-\frac{i\Delta t}{2}\lambda_r\left(\sum_s \lambda'^{(r)}_s n_s\right)^2} \leftrightarrow e^{-\frac{i\Delta t}{2}\lambda_r\left(\sum_s \lambda'^{(r)}_s \ket{1}\bra{1}_s\right)^2}
\end{equation}

As one may expect, this operator does the same thing on a qubit computational basis state $\ket{\vec{x}}$ that $U_A^{(r)}$ does on a Fock state, which is the phase rotation.

\begin{equation}
    e^{-\frac{i\Delta t}{2}\lambda_r\left(\sum_s \lambda'^{(r)}_s \ket{1}\bra{1}_s\right)^2}\ket{\vec{x}} = e^{-\frac{i\Delta t}{2}\lambda_r\left(\sum_s \lambda'^{(r)}_s x_s\right)^2}\ket{\vec{x}}
\end{equation}

We want to find an efficient way to approximately simulate $U_A^{(r)}$. That is, we want to find an operator $\tilde{U}_A^{(r)}$ that applies the specified phase rotation on $\ket{\vec{x}}$ as precisely as possible and with as few gates as possible. As a reminder, the error (and therefore precision) is measured by the operator norm of the difference in the ideal and approximate operators. $\epsilon$ is a parameter that limits the maximum error of this approximation and it affects the cost of $\tilde{U}_A^{(r)}$. A tighter bound requires more gates to achieve that bound.

\begin{equation}
    \epsilon \geq ||U_A^{(r)} - \tilde{U}_A^{(r)}|| = ||e^{-\frac{i\Delta t}{2}\lambda_r\left(\sum_s \lambda'^{(r)}_s n_s\right)^2} - U_A^{(r)}|| \label{eq: error}
\end{equation}

\eqref{eq: brute} presented a method that simulated $U_A^{(r)}$ exactly (no error) but it required $O(n^2)$ gates. We now present a $\tilde{U}_A^{(r)}$ that requires $O(n(\log{\frac{n}{\epsilon}})^2)$ gates. The idea is to coherently calculate the phase of the phase rotation on a separate qubit register and use these qubits representing the phase as controls to apply a phase rotation.

More formally, denote the value of the summation in the phase rotation as $y_r(\vec{x})$.

\begin{equation}
    \begin{split}
        y_r(\vec{x}) &= \sum_s \lambda'^{(r)}_s x_s \\
        U_A^{(r)}\ket{\vec{x}} &= e^{-i\frac{\Delta t}{2}\lambda_ry_r(\vec{x})^2}\ket{\vec{x}}
    \end{split}
\end{equation}

We begin with a register of $n$ qubits in a Fock state $\ket{\vec{x}}$ and introduce an ancilla register of $\frac{m}{2}$ and another of $m$ qubits, both initalized to zeros. We calculate an approximation $\tilde{y}_r(\vec{x})$ onto the smaller ancilla register. We use the value stored in the smaller register to compute an approximation $\tilde{y}_r(\vec{x})^2$ into the larger register. Then we use values of the $m$ ancilla qubits to rotate the phase by $e^{-i\frac{\Delta t}{2}\lambda_r\tilde{y}_r(\vec{x})^2}\ket{\vec{x}}$. Finally, we uncompute the ancilla register and remove them afterwards. These steps comprise the proposed $\tilde{U}_A^{(r)}$.

\begin{align}
    \ket{\vec{x}} &\rightarrow \ket{\vec{x}}\ket{0^{\frac{m}{2}}}\ket{0^m} \\
    &\rightarrow \ket{\vec{x}}\ket{\tilde{y}_r(\vec{x})}\ket{0^m} \label{eq: summation}\\
    &\rightarrow \ket{\vec{x}}\ket{\tilde{y}_r(\vec{x})}\ket{\tilde{y}_r(\vec{x})^2} \label{eq: squaring} \\
    &\rightarrow e^{-i\frac{\Delta t}{2}\lambda_r\tilde{y}_r(\vec{x})^2}\ket{\vec{x}}\ket{\tilde{y}_r(\vec{x})}\ket{\tilde{y}_r(\vec{x})^2} \label{eq: phase} \\
    &\rightarrow e^{-i\frac{\Delta t}{2}\lambda_r\tilde{y}_r(\vec{x})^2}\ket{\vec{x}}\ket{0^{\frac{m}{2}}}\ket{0^m} \\
    &\rightarrow e^{-i\frac{\Delta t}{2}\lambda_r\tilde{y}_r(\vec{x})^2}\ket{\vec{x}} \\
    &= \tilde{U}_A^{(r)}\ket{\vec{x}}
\end{align}

\section{Summation}

Because $\lambda'^{(r)}$ can assume any real (including irrational) values, a finite set of qubits will not be able to represent $y_r(\vec{x}) = \sum_s \lambda'^{(r)}_s x_s$ precisely. Instead, we compute an approximation $\tilde{y}_r(\vec{x})$ that can be expressed in $\frac{m}{2}$ bits. We essentially truncate (or round) each term in the sum and add them together. $m$ can be increased to enhance precision at the cost of ancilla qubits and more gates, and vice versa. In short, we'd like to use the qubits $\ket{\vec{x}}$ to transform $\ket{0^{\frac{m}{2}}}$ to $\ket{\tilde{y}_r(\vec{x})}$.

\begin{equation}
    \ket{\vec{x}}\ket{0^{\frac{m}{2}}} \rightarrow \ket{\vec{x}}\ket{\tilde{y}_r(\vec{x})} = \ket{\vec{x}}\ket{\sum_s \tilde{\lambda}'^{(r)}_s x_s}
    \label{eq: summation2}
\end{equation}

The encoding between values $z$ and qubit states must be able to handle negative numbers and must be able to hold the all possible values of $\tilde{y}_r(\vec{x})$. We define $Y_r$ as the largest number (by absolute value) that the register must hold, and we assign qubit $j$ to represent a value $2^{M_r - (\frac{m}{2} - 1) + j}$, where $M_r$ is defined below. We use a signed method, so qubit $\frac{m}{2} - 1$ represents the sign of the number. 

\begin{equation}
    Y_r = \max_{\vec{x} \in \{0, 1\}^n} |y_r(\vec{x})| = \max\left(\sum_{s : \lambda^{(r)}_s > 0} |\lambda^{(r)}_s|, \sum_{s : \lambda^{(r)}_s < 0} |\lambda^{(r)}_s|\right)
\end{equation}
\begin{equation}
    M_r = \lfloor \log_2{Y_r} \rfloor + 2
\end{equation}
\begin{equation}
    \ket{\vec{y}} \leftrightarrow \sum_{j = -\infty}^{\frac{m}{2} - 1} (-1)^{\delta_{j, \frac{m}{2} - 1}}y_j2^{M_r - (\frac{m}{2} - 1) + j}
\end{equation}

Then this register can hold all values between $-2^{M_r}$ and $2^{M_r} - 2^{M_r - (\frac{m}{2} - 1)}$ that are multiples of $2^{M_r - (\frac{m}{2} - 1)}$. By the definition of $M_r$, all possible $\tilde{y}_r(\vec{x})$ lie within this range. We can then define the bits $[\lambda^{(r)}_s]_j$ of each $\lambda^{(r)}_s$ in this signed binary form.

\begin{equation}
    \lambda^{(r)}_s = \sum_{j = -\infty}^{\frac{m}{2} - 1} (-1)^{\delta_{j, \frac{m}{2} - 1}}[\lambda^{(r)}_s]_j2^{M_r - (\frac{m}{2} - 1) + j}
\end{equation}

Since we only care about the bits for the values $2^{M_r - (\frac{m}{2} - 1)}$ to $2^{M_r}$, $\tilde{\lambda}^{(r)}_s$ is the accordingly truncated version of $\lambda^{(r)}_s$.

\begin{equation}
    \tilde{\lambda}^{(r)}_s = \sum_{j = 0}^{\frac{m}{2} - 1} (-1)^{\delta_{j, \frac{m}{2} - 1}}[\lambda^{(r)}_s]_j2^{M_r - (\frac{m}{2} - 1) + j}
\end{equation}

As shown in \ref{fig: U_A}, \eqref{eq: summation} can be achieved by a series of controlled additions. There are many ways to implement such an adder, we chose the most direct way, which simply converts a classical adder into quantum circuits. \ref{fig: add} shows how to implement a controlled addition of an arbitrary $w$ in the described signed form.

\begin{equation}
    w = \sum_{j = 0}^{\frac{m}{2} - 1} (-1)^{\delta_{j, \frac{m}{2} - 1}}[w]_j2^{M_r - (\frac{m}{2} - 1) + j}
\end{equation}

\section{Squaring}

We now have a register of $\frac{m}{2}$ qubits representing the value $\tilde{y}_r(\vec{x})$ and use this to calculate a value of $\tilde{y}_r(\vec{x})^2$ on the register of $m$ qubits.

\begin{equation}
    \ket{\vec{x}}\ket{\tilde{y}_r(\vec{x})}\ket{0^m} \rightarrow \ket{\vec{x}}\ket{\tilde{y}_r(\vec{x})}\ket{\tilde{y}_r(\vec{x})^2} \label{eq: y^2}
\end{equation}

We now design a circuit that adds the square of an arbitrary signed $\frac{m}{2}$-qubit $\ket{w}$ to an m-qubit register.

\begin{equation}
    \ket{w}\ket{z} \rightarrow \ket{w}\ket{z + w^2}
\end{equation}

Because the smallest value represented by a qubit in the $\ket{w}$ register is $2^{M_r - (\frac{m}{2} - 1)}$, the smallest value represented by a qubit in the $\ket{z}$ register will be $2^{2(M_r - (\frac{m}{2} - 1))}$. So qubit $k$ in the latter register will have value $(-1)^{\delta_{k, m - 1}}2^{2(M_r - (\frac{m}{2} - 1)) + k}$.

We can expand one of the $w$ to turn $w^2$ into a sum.

\begin{equation}
    w^2 = \sum_{j = 0}^{\frac{m}{2} - 1} (-1)^{\delta_{j, \frac{m}{2} - 1}}[w]_j2^{M_r - (\frac{m}{2} - 1) + j}w \label{eq: w^2}
\end{equation}

If one is to add $w^2$ to $z$, it's equivalent to adding each of these terms separately. The $j$th term in \eqref{eq: w^2} is only nonzero if $[w]_j = 1$, so it's equivalent to a having a control on the $j$th qubit in $\ket{w}$. Adding $2^{M_r - (\frac{m}{2} - 1) + j}w$ is equivalent to adding $w$ but shifted up $j$ qubits. Note that because $[w]_{\frac{m}{2} - 1}$ is a sign bit, it must continue to be added to all bits in $\ket{z}$ until the end, accordingly with classical signed addition. \ref{fig: squaring} shows a quantum circuit that adds $[w]_j2^{M_r - (\frac{m}{2} - 1) + j}w$.
We can handle the $(-1)^{\delta_{j, \frac{m}{2} - 1}}$, which flips the sign for $j = \frac{m}{2} - 1$, by performing a controlled subtraction instead of an addition. A controlled subtraction is equivalent to the Hermitian (in this case, a reverse circuit) of the controlled addition.

Performing the circuit described in \ref{eq: w^2} using $w = \tilde{y}_r(\vec{x})$ will successfully accomplish \ref{eq: y^2}.

\section{Phase Rotation}

We now have a register of $m$ qubits in which the $k$th qubit has state $\ket{[\tilde{y}_r(\vec{x})^2]_k}$.

\begin{equation}
    \tilde{y}_r(\vec{x})^2 = \sum_k (-1)^{\delta_{k, m - 1}}[\tilde{y}_r(\vec{x})^2]_k2^{2(M_r - (\frac{m}{2} - 1)) + k}
\end{equation}

As described in \eqref{eq: phase}, we'd like to apply a phase of $e^{-i\frac{\Delta t}{2}\lambda_r\tilde{y}_r(\vec{x})^2}$.

\begin{equation}
    \begin{split}
        \ket{\vec{x}}\ket{\tilde{y}_r(\vec{x})}\ket{\tilde{y}_r(\vec{x})^2} &\rightarrow e^{-i\frac{\Delta t}{2}\lambda_r\tilde{y}_r(\vec{x})^2}\ket{\vec{x}}\ket{\tilde{y}_r(\vec{x})}\ket{\tilde{y}_r(\vec{x})^2} \\
        &= e^{-i\frac{\Delta t}{2}\lambda_r\sum_k (-1)^{\delta_{k, m - 1}}[\tilde{y}_r(\vec{x})^2]_k2^{2(M_r - (\frac{m}{2} - 1)) + k}}\ket{\vec{x}}\ket{\tilde{y}_r(\vec{x})}\ket{\tilde{y}_r(\vec{x})^2}\\
        &= \ket{\vec{x}}\ket{\tilde{y}_r(\vec{x})}\prod_k e^{-i[\tilde{y}_r(\vec{x})^2]_k(-1)^{\delta_{k, m - 1}}\frac{\Delta t}{2}\lambda_r2^{2(M_r - (\frac{m}{2} - 1)) + k}}\ket{\tilde{y}_r(\vec{x})^2}
    \end{split}
\end{equation}

The individual phase rotation $e^{-i[\tilde{y}_r(\vec{x})^2]_k(-1)^{\delta_{k, m - 1}}\frac{\Delta t}{2}\lambda_r2^{2(M_r - (\frac{m}{2} - 1)) + k}}$ can be applied with a phase gate $P(\varphi)$ on qubit $k$ with $\varphi = -(-1)^{\delta_{k, m - 1}}\frac{\Delta t}{2}\lambda_r2^{2(M_r - (\frac{m}{2} - 1)) + k}$. These gates are shown in \ref{fig: U_A}.

\begin{equation}
    P(\varphi) = \begin{bmatrix}
        1 & 0 \\ 0 & e^{i\varphi}
    \end{bmatrix} = \ket{0}\bra{0} + e^{i\varphi}\ket{1}\bra{1}
\end{equation}

Once the phase has been applied, it is necessary to uncompute the values $\tilde{y}_r(\vec{x}), \tilde{y}_r(\vec{x})^2$ computed in the ancilla registers and reset the registers to zeros. One can simply do this by applying all the previous gates (in the summation and squaring) in reverse order. Then the empty registers can be safely removed and $\tilde{U}_A^{(r)}$ has successfully been applied.

\section{Complexity Analysis}

The controlled constant addition in \ref{fig: add} requires $\frac{m}{2} - 1$ ancilla qubits (in addition to the target register) and its gate count depends on the values of $[w]_j$. At maximum, the gate count is $1.5m - 5$ Toffolis and $2.5m - 7$ CNOTs.

Next, we wish to establish an upper bound on the number of gates in the circuit in \ref{fig: squaring}. Observe that for each $\ket{z}_{j + 2}, ..., \ket{z}_{m - 1}$, there are $8$ Toffolis \footnote{When iterating through each of the qubits in $\ket{w}$, inevitably one or more of these qubits will be $\ket{w}_j$. One of the gates in an iteration is a Toffoli controlled by $\ket{w}_j$ and the iterated qubit, which in this case is itself. Then that Toffoli will actually be a CNOT with a lower cost. We ignore this edge case because we only need an upper bound}. There are $4$ additional Toffolis ($2$ for the highest bit and $2$ at the beginning and end). This is a total of $8(m - j - 2) + 4$ Toffolis in \ref{fig: squaring}.

Since the squaring protocol performs \ref{fig: squaring} for each $j \in [0, \frac{m}{2})$, we can calculate the total number of Toffolis in the squaring protocol.

\begin{equation}
    \begin{split}
        \sum_{j = 0}^{\frac{m}{2} - 1} (8(m - j - 2) + 4) &= 8m^2 - 8\frac{\frac{m}{2}(\frac{m}{2} - 1)}{2} - 8m + 4m \\
        &= 7m^2 - 2m
    \end{split}
\end{equation}

In our implementation of $\tilde{U}_A^{(r)}$, there are $2n$ controlled constant additions ($n$ to compute and $n$ to uncompute) and $2$ squaring protocols (once to compute and uncompute). The phase rotations only contribute $n$ single-qubit gates, which are negligible. We can calculate the total number of Toffolis in $\tilde{U}_A^{(r)}$.

\begin{equation}
    \begin{split}
        \tilde{N}_{CCNOT} &= 2n(1.5m - 5) + 2(7m^2 - 2m) \\
        &= 3mn - 10n + 14m^2 - 4m \\
        &= O(mn + m^2) \label{eq: toffolis}
    \end{split}
\end{equation}

A Toffoli gate can be constructed from $2$ Hadamard gates (negligible), $6$ CNOT gates, and $7$ T-gates. There are also $2.5mn - 7n$ additional CNOTs from the controlled constant additions.

\begin{equation}
    \begin{split}
        \tilde{N}_{CNOT} &= 6\tilde{N}_{CCNOT} + 2.5mn - 7n \\
        &= 20.5mn - 67n + 84m^2 - 24m \\
        &= O(mn + m^2) \label{eq: cnots}
    \end{split}
\end{equation}

\begin{equation}
    \begin{split}
        \tilde{N}_{T} &= 7\tilde{N}_{CCNOT} \\
        &= 21mn - 70n + 98m^2 - 28m \\
        &= O(mn + m^2) \label{eq: Ts}
    \end{split}
\end{equation}

\section{Error Analysis}

The size of the ancilla registers is determined by $m$ and affects the precision of the applied phase. A larger $m$ means more ancilla registers and a more precise phase, but quadratically increases the gate cost according to \eqref{eq: toffolis}. If we want to guarantee an error below $\epsilon$ accordingly with \eqref{eq: error}, $m$ must be sufficiently large. We now calculate the relation between $\epsilon$ and $m$. This is to say that we upper bound the error $||U_A^{(r)} - \tilde{U}_A^{(r)}||$ in terms of $m$.

The error originates from the fact that we need to truncate our representation of $\lambda^{(r)}_s$ into $\tilde{\lambda}^{(r)}_s$, and propagates into $\tilde{U}_A^{(r)}$. So we begin by bounding the error in $\tilde{\lambda}^{(r)}_s$.

\begin{equation}
    \begin{split}
        |\lambda^{(r)}_s - \tilde{\lambda}^{(r)}_s| = \sum_{j = -\infty}^{-1} [\lambda^{(r)}_s]_j2^{M_r - (\frac{m}{2} - 1) + j} \leq \sum_{j = -\infty}^{-1} 2^{M_r - (\frac{m}{2} - 1) + j} \leq 2^{M_r - (\frac{m}{2} - 1)} \leq \frac{4Y_r}{2^{\frac{m}{2} - 1}}
    \end{split}
\end{equation}

We propagate this error to $\tilde{y}(\vec{x})$.

\begin{equation}
    \begin{split}
        |y_r(\vec{x}) - \tilde{y}(\vec{x})| = |\sum_s (\lambda^{(r)}_s - \tilde{\lambda}^{(r)}_s)x_s| \leq n|\lambda^{(r)}_s - \tilde{\lambda}^{(r)}_s| \leq \frac{Y_r}{2^{\frac{m}{2} - 3}}
    \end{split}
\end{equation}

We propagate this error to $\tilde{y}(\vec{x})^2$.

\begin{equation}
    \begin{split}
        |y_r(\vec{x})^2 - \tilde{y}(\vec{x})^2| &= |2y_r(\vec{x})(y_r(\vec{x}) - \tilde{y}(\vec{x})) - (y_r(\vec{x}) - \tilde{y}(\vec{x}))^2| \\
        &\leq \max\left(2|y_r(\vec{x})|n\frac{Y_r}{2^{\frac{m}{2} - 3}}, \left(n\frac{Y_r}{2^{\frac{m}{2} - 3}}\right)^2\right)
    \end{split}
\end{equation}

\begin{equation}
    \begin{split}
        ||U_A^{(r)} - \tilde{U}_A^{(r)}|| &= \max_{\ket{\psi} : ||\ket{\psi}|| = 1} ||(U_A^{(r)} - \tilde{U}_A^{(r)})\ket{\psi}|| \\
        &= \max_{\vec{x} \in \{0, 1\}^n} ||(U_A^{(r)} - \tilde{U}_A^{(r)})\ket{\vec{x}}|| \\
        &= \max_{\vec{x} \in \{0, 1\}^n} ||(e^{-i\frac{\Delta t}{2}\lambda_ry_r(\vec{x})^2} - e^{-i\frac{\Delta t}{2}\lambda_r\tilde{y}_r(\vec{x})^2})\ket{\vec{x}}|| \\
        &= \max_{\vec{x} \in \{0, 1\}^n} ||(e^{-i\frac{\Delta t}{2}|\lambda_r|(y_r(\vec{x})^2 - \tilde{y}_r(\vec{x})^2)} - 1)\ket{\vec{x}}|| \\
        &\leq \max_{\vec{x} \in \{0, 1\}^n} \frac{\Delta t}{2}|\lambda_r||y_r(\vec{x})^2 - \tilde{y}_r(\vec{x})^2| \\
        &\leq \frac{\Delta t}{2}|\lambda_r|\max_{\vec{x} \in \{0, 1\}^n} \max\left(2|y_r(\vec{x})|n\frac{Y_r}{2^{\frac{m}{2} - 3}}, \left(n\frac{Y_r}{2^{\frac{m}{2} - 3}}\right)^2\right) \\
        &= n\frac{|\lambda_r| Y_r^2\Delta t}{2^{\frac{m}{2} - 4}}\max\left(1, \frac{n}{2^{\frac{m}{2} - 2}}\right)
    \end{split}
\end{equation}

We want $m$ to be large enough that this quantity is at most $\epsilon$.

\begin{equation}
    \begin{split}
        m_{min} &= \max(2\log_2\left(n\frac{|\lambda_r|Y_r^2\Delta t}{\epsilon}\right) + 3, \log_2\left(n^2\frac{|\lambda_r|Y_r^2\Delta t}{\epsilon}\right) + 5) \\
        &\geq 2\log_2(n) + 2\log_2\left(\frac{|\lambda_r|Y_r^2\Delta t}{\epsilon}\right) + 3
    \end{split}
\end{equation}

Clearly $m_{min}$ is asymptotically logarithmic in $n$, as long as $|\lambda_r|$ and $Y_r$ are not exponential in $n$ (in reality they are sublinear). Referring back to \eqref{eq: toffolis}, the total asymptotic gate cost of $\tilde{U}_A^{(r)}$ is therefore $O(mn + m^2) = O(n\log{n} + (\log{n})^2) = O(n\log{n})$.

$m$ is also logarithmic in $\Delta t$ and in $\frac{1}{\epsilon}$.

\section{Numerics}

Although the gate cost of this $\tilde{U}_A^{(r)}$ is asympotically lower than the quadratic cost of \eqref{eq: brute}, it's useful to get a (very rough) numerical estimate of the gate counts to understand the scale ($n$) at which it actually becomes a better option than alternatives.

The value of $\Delta t$ depends on the particular Trotter-like algorithm used. A very loose upper bound for $\Delta t$ is $t$ itself. \cite{Gate_Count} uses $t = 6000 \frac{1}{E_h}$ and $\epsilon = 10^{-6}$. We decomposed Hamiltonians of hydrogen chain systems for various $n$ ($n$ is the number of $H$s and the number of orbitals) to obtain values of $|\lambda_r|$ and $Y_r$. It turns out for these systems that $|\lambda_r|Y_r^2 \approx n\log_2{n} E_h$.

\begin{equation}
    \begin{split}
        m_{min} &\leq 2\log_2(n) + 2\log_2\left(\frac{|\lambda_r|Y_r^2 t}{\epsilon}\right) + 3 \\
        &\leq 2\log_2(n) + 2\log_2\left(\frac{n\log_2{n}\times 6000}{10^{-6}}\right) + 3 \\
        &\approx 4\log_2(n) + 2\log_2(\log_2(n)) + 68
    \end{split}
\end{equation}

In practice, $\Delta << t$, so $t$ is not a useful order of magnitude for $\Delta t$. \cite{Gate_Count} also mentions a value $\Delta t = 0.01 \frac{1}{E_h}$ which may give a more accurate sense of the scale of $\Delta t$.

\begin{equation}
    \begin{split}
        m_{min} \leq 2\log_2(n) + 2\log_2\left(\frac{n\log_2{n}\times 0.01}{10^{-6}}\right) + 3 \approx 4\log_2(n) + 2\log_2(\log_2(n)) + 17
    \end{split}
\end{equation}

In non-fault-tolerant computing, quantum gate cost is often measured by the number of two-qubit gates (in our case, CNOTs). So we can substitute this function $m(n)$ back into \eqref{eq: cnots} and plot $\tilde{N}_{CNOT}$ as a function of $n$. We can then compare $\tilde{N}_{CNOT}$ with $N_{CNOT} = 2{n \choose 2}$, the number of CNOTs necessary to implement \eqref{eq: brute} (a controlled Z-rotation can be done with $2$ CNOTs and two $P$ gates) \cite{gates}. It turns out that $\tilde{N}_{CNOT} = N_{CNOT}$ when $n = 3023.3$, above which $\tilde{N}_{CNOT} < N_{CNOT}$.

\textcolor{red}{I will add this plot later}

In fault-tolerant computing, the bottleneck is the number of T-gates. In this framework, only Clifford gates and T gates are allowed, and gates not in this category, such as $P$ gates, must be approximated by decomposing it into Clifford and T gates. \textcolor{red}{I need to ask Aram how exactly this happens and how I should calculate this. Because there are P gates in both methods. Also there was that thing about the log(1/eps) error which I didn't exactly catch.}


%When is O(n log n) really better than n^2/2?

%Mention that if it's really diagonal, then we can just qdrift the rotations.

\section{Discussion}

The asymptotic advantage of the coherent method ($O(n^3\log_2{n})$) over the naive method ($O(n^4)$) only results in an overall asymptotic advantage if $U(R^{(r)})$ can also be implemented in with sub-quadratic gate complexity. \cite{FSN} decomposed $U(R^{(r)})$ into a series of at most ${n \choose 2}$ two-orbital basis rotations, each implementable in a constant number of quantum gates. However, for certain types of Hamiltonians, the necessary number of such rotations is often much smaller.

The minimum number of rotation is $0$, which occurs when $R^{(r)} = I$; that is, when $Q^{(r)}$ is already diagonal. However, if this is the case for all $r \in [0, n^2)$ in $\hat{H}_{2e}$, then all of the terms in $\hat{H}_{2e}$ can be combined into $n^2$ terms as shown in \eqref{eq: collapse}, resulting in a $O(n^2)$ complexity for $\hat{H}_{2e}$, in which case the coherent method would not be useful.

\begin{equation}
    \begin{split}
        \hat{H}_{2e} &= \frac{1}{2}\sum_r\lambda_r\left(\sum_{i,j} Q^{(r)}_{i, j}a^\dag_ia_j\right)^2 \\
        &= \frac{1}{2}\sum_r\lambda_r(\sum_{s} \lambda'^{(r)}_sn_s)^2 \\
        &= \frac{1}{2}\sum_{s, s'} n_sn_{s'}\sum_r\lambda_r\lambda'^{(r)}_s\lambda'^{(r)}_{s'}
    \end{split}
    \label{eq: collapse}
\end{equation}

In fact, more generally, if $R^{(r)}$ is invariant across $r$, all of the different $\tilde{n}^{(r)}_s$ are in the same basis across $r$. Then the basis rotations $U(R^{(r)})$ and $U(R^{(r)})^\dag$ can be performed at the beginning and end without the need to switch bases between $r$. Then $\hat{H}_{2e}$ combines into $O(n^2)$ terms in \eqref{eq: collapse2}, just like in \eqref{eq: collapse}. Since $U(R^{(r)})$ uses $O(n^2)$ but is only performed twice, the total complexity is $O(n^2)$ for $\hat{H}_{2e}$, and the coherent would not be useful in this case either.

\begin{equation}
    \begin{split}
        \hat{H}_{2e} &= \frac{1}{2}\sum_r\lambda_r\left(\sum_s \lambda'^{(r)}_s \tilde{n}_s\right)^2 \\
        &= \frac{1}{2}\sum_{s, s'} \tilde{n}_s\tilde{n}_{s'}\sum_r\lambda_r\lambda'^{(r)}_s\lambda'^{(r)}_{s'}
    \end{split}
    \label{eq: collapse2}
\end{equation}

Moreover, if all the $Q^{(r)}$ are diagonally dominated (the magnitude of the terms on its diagonal dominates over the non-diagonal terms) or if they have very few ($O(\sqrt{n})$) nonzero off-diagonal elements, then the expansion of $\hat{H}_{2e}$ also contains a reduced number of terms that make the coherent method unnecessary.

For the coherent method to be useful, all of the $U(R^{(r)})$ should decompose to $o(n^2)$ rotations, but $Q^{(r)}$ should have $\omega(\sqrt{n})$ non-zero terms (not a tight bound). This opens an area for future work to address: identifying useful Hamiltonians that satisfy these conditions.

In the next section, we will focus on one family of Hamiltonians that satisfy these conditions.

%We need it to be O(n) rotations but O(n^2) nonzeros in R.

%Example? k-local.


%%%%%%%%%%%%%%%% end table %%%%%%%%%%%%%%%%%%% 


% From mitthesis package
% Version: 1.07, 2024/09/26
% Documentation: https://ctan.org/pkg/mitthesis


\chapter{Convex Optimization of Decomposition of Electron-Electron Interaction Hamiltonian Term}

\section{Decomposition of Hamiltonian Term}
\section{Gradient Descent}
\section{Gate Count Analysis}
\section{Numerics}


%%%%%%%%%%%%%%%% end table %%%%%%%%%%%%%%%%%%% 


% From mitthesis package
% Version: 1.07, 2024/09/26
% Documentation: https://ctan.org/pkg/mitthesis


\chapter{Conclusion}

In this work, we have designed in detail a quantum circuit that reduces the asymptotic gate complexity of simulating certain types of electronic structure Hamiltonians, in particular those that are low-range. With specific calculations of $T$ counts and error bounds, we showed that asymptotic advantages of the proposed ``double-factorized coherent method'' over the competing ``double-factorized expansion method'' begin to manifest at a fairly low threshold of $n$: for example, fewer than a hundred orbitals for hydrogen chain systems.

Furthermore, we proposed a framework for adapting the usage of the double-factorized coherent method from strictly-low-range electronic structure Hamiltonians to those that are ``low-range dominant'', which tend to be the case for practical systems. We intend to continue formulating more specific decomposition techniques in this framework and numerically analyze the performance of the coherent method in conjunction with qDRIFT and qDRIFT-like methods; unfortunately this was not feasible within the timeframe of this thesis. We strongly encourage this framework as an area of future work for others inspired by the findings in this paper. We also hope that our work opens up a search for other types of electronic structure Hamiltonians (beyond low-range) for which the asymptotic advantages of the coherent method can take effect, namely those with ``easy'' basis rotations.

%%%%%%%%%%%%%%%% end table %%%%%%%%%%%%%%%%%%% 


% % From mitthesis package
% Version: 1.07, 2024/09/26
% Documentation: https://ctan.org/pkg/mitthesis


\chapter{Convex Optimization of Decomposition of Electron-Electron Interaction Hamiltonian Term}

\section{Decomposition of Hamiltonian Term}
\section{Gradient Descent}
\section{Gate Count Analysis}
\section{Numerics}


%%%%%%%%%%%%%%%% end table %%%%%%%%%%%%%%%%%%% 


% % From mitthesis package
% Version: 1.07, 2024/09/26
% Documentation: https://ctan.org/pkg/mitthesis


\chapter{Conclusion}

In this work, we have designed in detail a quantum circuit that reduces the asymptotic gate complexity of simulating certain types of electronic structure Hamiltonians, in particular those that are low-range. With specific calculations of $T$ counts and error bounds, we showed that asymptotic advantages of the proposed ``double-factorized coherent method'' over the competing ``double-factorized expansion method'' begin to manifest at a fairly low threshold of $n$: for example, fewer than a hundred orbitals for hydrogen chain systems.

Furthermore, we proposed a framework for adapting the usage of the double-factorized coherent method from strictly-low-range electronic structure Hamiltonians to those that are ``low-range dominant'', which tend to be the case for practical systems. We intend to continue formulating more specific decomposition techniques in this framework and numerically analyze the performance of the coherent method in conjunction with qDRIFT and qDRIFT-like methods; unfortunately this was not feasible within the timeframe of this thesis. We strongly encourage this framework as an area of future work for others inspired by the findings in this paper. We also hope that our work opens up a search for other types of electronic structure Hamiltonians (beyond low-range) for which the asymptotic advantages of the coherent method can take effect, namely those with ``easy'' basis rotations.

%%%%%%%%%%%%%%%% end table %%%%%%%%%%%%%%%%%%% 




%%% Appendicies of thesis  %%%%%%%%%%%%%%%%%%%%%%%%%%%%%%%%%%%%%%%%%%%%%%%%%%%%%%%%%%%%%%%%%%%%%%%%

\appendix
% From mitthesis package
% Version: 1.01, 2023/07/04
% Documentation: https://ctan.org/pkg/mitthesis


\chapter{Quantum Circuit Diagrams}

\begin{figure}[h]
    \centering
    \Qcircuit @C=1em @R=1.6em {
        \lstick{} & \ctrl{4} & \qw      & \qw    & \qw      & \qw & \qw & \qw & \\
        \lstick{} & \qw      & \ctrl{3} & \qw    & \qw      & \qw & \qw & \qw & \\
        \lstick{} &          &          & \cdots &          &     &     &     & \\
        \lstick{} & \qw      & \qw      & \qw    & \ctrl{1} & \qw & \qw & \qw
        \inputgroupv{1}{4}{.8em}{2.4em}{\ket{\psi}}\\
        \lstick{} & \multigate{3}{+\lambda^{(r)}_0} & \multigate{3}{+\lambda^{(r)}_1} & \qw    & \multigate{3}{+\lambda^{(r)}_{n - 1}} & \multigate{8}{\substack{\ket{w}\ket{z} \rightarrow \\ \ket{w}\ket{z + w^2}}} & \qw & \qw \\
        \lstick{} & \ghost{+\lambda^{(r)}_0}       & \ghost{+\lambda^{(r)}_1}         & \qw    & \ghost{+\lambda^{(r)}_{n - 1}}        &  \ghost{\substack{\ket{w}\ket{z} \rightarrow \\ \ket{w}\ket{z + w^2}}}         & \qw & \qw \\
        \lstick{} & \nghost{+\lambda^{(r)}_0}      & \nghost{+\lambda^{(r)}_1}        & \cdots & \nghost{+\lambda^{(r)}_{n - 1}}       & \nghost{\substack{\ket{w}\ket{z} \rightarrow \\ \ket{w}\ket{z + w^2}}}     &     & \\
        \lstick{} & \ghost{+\lambda^{(r)}_0}       & \ghost{+\lambda^{(r)}_1}         & \qw    & \ghost{+\lambda^{(r)}_{n - 1}}        &  \ghost{\substack{\ket{w}\ket{z} \rightarrow \\ \ket{w}\ket{z + w^2}}} & \qw & \qw 
        \inputgroupv{5}{8}{.8em}{2.4em}{\ket{0^{\frac{m}{2}}}}\\
        \lstick{} & \qw & \qw & \qw & \qw & \ghost{\substack{\ket{w}\ket{z} \rightarrow \\ \ket{w}\ket{z + w^2}}} & \gate{P(-\frac{\Delta t}{2}\lambda_r2^{2(M_r - (\frac{m}{2} - 1))})} & \qw \\
        \lstick{} & \qw & \qw & \qw & \qw & \ghost{\substack{\ket{w}\ket{z} \rightarrow \\ \ket{w}\ket{z + w^2}}} & \gate{P(-\frac{\Delta t}{2}\lambda_r2^{2(M_r - (\frac{m}{2} - 1)) + 1})} & \qw \\
        \lstick{} & & & \cdots & & \nghost{\substack{\ket{w}\ket{z} \rightarrow \\ \ket{w}\ket{z + w^2}}} & \cdots & \\
        \lstick{} & \qw & \qw & \qw & \qw & \ghost{\substack{\ket{w}\ket{z} \rightarrow \\ \ket{w}\ket{z + w^2}}} & \gate{P(-\frac{\Delta t}{2}\lambda_r2^{2M_r - 1})} & \qw \\
        \lstick{} & \qw & \qw & \qw & \qw & \ghost{\substack{\ket{w}\ket{z} \rightarrow \\ \ket{w}\ket{z + w^2}}} & \gate{P(\frac{\Delta t}{2}\lambda_r2^{2M_r + 1})} & \qw 
        \inputgroupv{9}{13}{.8em}{2.4em}{\ket{0^m}}
    }
    \caption{Quantum circuit diagram for $U_A^{(r)}$ using arithmetic circuits (uncomputing not shown)}
    \label{fig: U_A}
\end{figure}

\begin{figure}[h]
    \centering
    \Qcircuit @C=1em @R=1.6em {
        \lstick{\cdots}          & [w]_0 & & \qquad [w]_1 & & & \qquad [w]_2 & & & [w]_3 & \qquad [w]_2 & & & & [w]_2 & \qquad [w]_1 & & & & [w]_1 & [w]_0 \\
        \lstick{\ket{\vec{x}}_s} & \qw      & \qw      & \qw      & \qw      & \qw      & \qw      & \qw      & \ctrl{7} & \ctrl{8}
                                 & \qw      & \qw      & \qw      & \ctrl{5} & \ctrl{6} & \qw      & \qw      & \qw      & \ctrl{3} & \ctrl{4} & \qw      & \qw \\
        \lstick{\cdots}          & \\
        \lstick{\ket{z}_0}       & \ctrl{1} & \qw      & \qw      & \qw      & \qw      & \qw      & \qw      & \qw      & \qw      
                                 & \qw      & \qw      & \qw      & \qw      & \qw      & \qw      & \qw      & \qw      & \qw      & \qw      & \ctrl{1} & \qw \\
        \lstick{\ket{0}}         & \targ    & \ctrl{1} & \ctrl{2} & \qw      & \qw      & \qw      & \qw      & \qw      & \qw 
                                 & \qw      & \qw      & \qw      & \qw      & \qw      & \qw      & \ctrl{2} & \ctrl{1} & \ctrl{1} & \qw      & \targ    & \qw \\
        \lstick{\ket{z}_1}       & \qw      & \ctrl{1} & \qw      & \ctrl{1} & \qw      & \qw      & \qw      & \qw      & \qw 
                                 & \qw      & \qw      & \qw      & \qw      & \qw      & \ctrl{1} & \qw      & \ctrl{1} & \targ    & \targ    & \qw      & \qw \\
        \lstick{\ket{0}}         & \qw      & \targ    & \targ    & \targ    & \ctrl{1} & \ctrl{2} & \qw      & \qw      & \qw
                                 & \qw      & \ctrl{2} & \ctrl{1} & \ctrl{1} & \qw      & \targ    & \targ    & \targ    & \qw      & \qw      & \qw      & \qw \\
        \lstick{\ket{z}_2}       & \qw      & \qw      & \qw      & \qw      & \ctrl{1} & \qw      & \ctrl{1} & \qw      & \qw
                                 & \ctrl{1} & \qw      & \ctrl{1} & \targ    & \targ    & \qw      & \qw      & \qw      & \qw      & \qw      & \qw      & \qw \\
        \lstick{\ket{0}}         & \qw      & \qw      & \qw      & \qw      & \targ    & \targ    & \targ    & \ctrl{1} & \qw
                                 & \targ    & \targ    & \targ    & \qw      & \qw      & \qw      & \qw      & \qw      & \qw      & \qw      & \qw      & \qw \\
        \lstick{\ket{z}_3}       & \qw      & \qw      & \qw      & \qw      & \qw      & \qw      & \qw      & \targ    & \targ
                                 & \qw      & \qw      & \qw      & \qw      & \qw      & \qw      & \qw      & \qw      & \qw      & \qw      & \qw      & \qw
        \gategroup{2}{2}{5}{2}{.7em}{--}
        \gategroup{2}{4}{7}{5}{.7em}{--}
        \gategroup{2}{7}{9}{8}{.7em}{--}
        \gategroup{2}{10}{10}{10}{.7em}{--}
        \gategroup{2}{11}{9}{12}{.7em}{--}
        \gategroup{2}{15}{8}{15}{.7em}{--}
        \gategroup{2}{16}{7}{17}{.7em}{--}
        \gategroup{2}{20}{6}{20}{.7em}{--}
        \gategroup{2}{21}{5}{21}{.7em}{--}
    }
    \caption{Example quantum circuit diagram for a controlled addition of $w$ for $\frac{m}{2} = 4$. The label $[w]_j$ indicates that the correspondingly boxed gates are only applied when $[w]_j = 1$. }
    \label{fig: add}
\end{figure}

\begin{figure}[h]
    \centering
    \Qcircuit @C=0.5em @R=1.2em {
        \lstick{\ket{w}_0}
            & \ctrl{6} & \qw      & \qw      & \qw      & \qw      & \qw      & \qw      & \qw      & \qw      & \qw      & \qw      & \qw      & \qw      & \qw      
            & \qw      & \qw      & \qw      & \qw      & \qw      & \qw      & \qw      & \qw      & \qw      & \qw      & \qw      & \qw      & \qw      & \qw      & \qw      & \qw      & \qw      & \ctrl{6} & \qw \\
        \lstick{\ket{w}_1}
            & \qw      & \qw      & \ctrl{6} & \ctrl{7} & \qw      & \qw      & \qw      & \qw      & \qw      & \qw      & \qw      & \qw      & \qw      & \qw      
            & \qw      & \qw      & \qw      & \qw      & \qw      & \qw      & \qw      & \qw      & \qw      & \qw      & \qw      & \qw      & \ctrl{7} & \ctrl{6} & \qw      & \qw      & \ctrl{2} & \qw      & \qw \\
        \lstick{\cdots}          & & & & & & & & & & & & & & & & & & & & & & & & & & & & & & & & \\
        \lstick{\ket{w}_j}
            & \qw      & \qw      & \qw      & \qw      & \qw      & \qw      & \qw      & \qw      & \qw      & \qw      & \qw      & \qw      & \ctrl{14}& \ctrl{2}  
            & \qw      & \qw      & \qw      & \ctrl{12}& \ctrl{2} & \qw      & \qw      & \qw      & \qw      & \ctrl{8} & \ctrl{2} & \qw      & \qw      & \qw      & \qw      & \ctrl{4} & \ctrl{5} & \qw      & \qw \\
        \lstick{\cdots}          & & & & & & & & & & & & & & & & & & & & & & & & & & & & & & & & \\
        \lstick{\ket{w}_{\frac{m}{2} - 1}}
            & \qw      & \qw      & \qw      & \qw      & \qw      & \qw      & \ctrl{6} & \ctrl{7} & \qw      & \qw      & \ctrl{10}& \ctrl{11}& \qw      & \ctrl{13}
            & \ctrl{11}& \ctrl{10}& \qw      & \qw      & \ctrl{11}& \qw      & \ctrl{7} & \ctrl{6} & \qw      & \qw      & \ctrl{7} & \qw      & \qw      & \qw      & \qw      & \qw      & \qw      & \qw      & \qw \\
        \lstick{\ket{z}_j}
            & \ctrl{1} & \qw      & \qw      & \qw      & \qw      & \qw      & \qw      & \qw      & \qw      & \qw      & \qw      & \qw      & \qw      & \qw      
            & \qw      & \qw      & \qw      & \qw      & \qw      & \qw      & \qw      & \qw      & \qw      & \qw      & \qw      & \qw      & \qw      & \qw      & \qw      & \qw      & \qw      & \ctrl{1} & \qw \\
        \lstick{\ket{0}} 
            & \targ    & \ctrl{1} & \ctrl{2} & \qw      & \qw      & \qw      & \qw      & \qw      & \qw      & \qw      & \qw      & \qw      & \qw      & \qw 
            & \qw      & \qw      & \qw      & \qw      & \qw      & \qw      & \qw      & \qw      & \qw      & \qw      & \qw      & \qw      & \qw      & \ctrl{2} & \ctrl{1} & \ctrl{1} & \qw      & \targ    & \qw \\
        \lstick{\ket{z}_{j + 1}}
            & \qw      & \ctrl{1} & \qw      & \ctrl{1} & \qw      & \qw      & \qw      & \qw      & \qw      & \qw      & \qw      & \qw      & \qw      & \qw 
            & \qw      & \qw      & \qw      & \qw      & \qw      & \qw      & \qw      & \qw      & \qw      & \qw      & \qw      & \qw      & \ctrl{1} & \qw      & \ctrl{1} & \targ    & \targ    & \qw      & \qw \\
        \lstick{\ket{0}}
            & \qw      & \targ    & \targ    & \targ    & \qw      & \qw      & \qw      & \qw      & \qw      & \qw      & \qw      & \qw      & \qw      & \qw
            & \qw      & \qw      & \qw      & \qw      & \qw      & \qw      & \qw      & \qw      & \qw      & \qw      & \qw      & \qw      & \targ    & \targ    & \targ    & \qw      & \qw      & \qw      & \qw \\
        \lstick{\cdots}          & & & & & \ddots & & & & & & & & & & & & & & & & & & & & & \iddots & & & & & \\
        \lstick{\ket{0}}
            & \qw      & \qw      & \qw      & \qw      & \qw      & \ctrl{1} & \ctrl{2} & \qw      & \qw      & \qw      & \qw      & \qw      & \qw      & \qw 
            & \qw      & \qw      & \qw      & \qw      & \qw      & \qw      & \qw      & \ctrl{2} & \ctrl{1} & \ctrl{1} & \qw      & \qw      & \qw      & \qw      & \qw      & \qw      & \qw      & \qw      & \qw \\
        \lstick{\ket{z}_{j + \frac{m}{2} - 1}}
            & \qw      & \qw      & \qw      & \qw      & \qw      & \ctrl{1} & \qw      & \ctrl{1} & \qw      & \qw      & \qw      & \qw      & \qw      & \qw 
            & \qw      & \qw      & \qw      & \qw      & \qw      & \qw      & \ctrl{1} & \qw      & \ctrl{1} & \targ    & \targ    & \qw      & \qw      & \qw      & \qw      & \qw      & \qw      & \qw      & \qw \\
        \lstick{\ket{0}}
            & \qw      & \qw      & \qw      & \qw      & \qw      & \targ    & \targ    & \targ    & \qw      & \qw      & \qw      & \qw      & \qw      & \qw
            & \qw      & \qw      & \qw      & \qw      & \qw      & \qw      & \targ    & \targ    & \targ    & \qw      & \qw      & \qw      & \qw      & \qw      & \qw      & \qw      & \qw      & \qw      & \qw \\
        \lstick{\cdots}          & & & & & & & & & \ddots & & & & & & & & & & & \iddots & & & & & & & & & & & \\
        \lstick{\ket{0}}
            & \qw      & \qw      & \qw      & \qw      & \qw      & \qw      & \qw      & \qw      & \qw      & \ctrl{1} & \ctrl{2} & \qw      & \qw      & \qw
            & \qw      & \ctrl{2} & \ctrl{1} & \ctrl{1} & \qw      & \qw      & \qw      & \qw      & \qw      & \qw      & \qw      & \qw      & \qw      & \qw      & \qw      & \qw      & \qw      & \qw      & \qw \\
        \lstick{\ket{z}_{m - 2}}
            & \qw      & \qw      & \qw      & \qw      & \qw      & \qw      & \qw      & \qw      & \qw      & \ctrl{1} & \qw      & \ctrl{1} & \qw      & \qw
            & \ctrl{1} & \qw      & \ctrl{1} & \targ    & \targ    & \qw      & \qw      & \qw      & \qw      & \qw      & \qw      & \qw      & \qw      & \qw      & \qw      & \qw      & \qw      & \qw      & \qw \\
        \lstick{\ket{0}}
            & \qw      & \qw      & \qw      & \qw      & \qw      & \qw      & \qw      & \qw      & \qw      & \targ    & \targ    & \targ    & \ctrl{1} & \qw
            & \targ    & \targ    & \targ    & \qw      & \qw      & \qw      & \qw      & \qw      & \qw      & \qw      & \qw      & \qw      & \qw      & \qw      & \qw      & \qw      & \qw      & \qw      & \qw \\
        \lstick{\ket{z}_{m - 1}}
            & \qw      & \qw      & \qw      & \qw      & \qw      & \qw      & \qw      & \qw      & \qw      & \qw      & \qw      & \qw      & \targ    & \targ
            & \qw      & \qw      & \qw      & \qw      & \qw      & \qw      & \qw      & \qw      & \qw      & \qw      & \qw      & \qw      & \qw      & \qw      & \qw      & \qw      & \qw      & \qw      & \qw
    }
    \caption{Quantum circuit diagram for $\ket{w}\ket{z} \rightarrow \ket{w}\ket{z + [w]_j2^{M_r - (\frac{m}{2} - 1) + j}w}$. This is essentially the equivalent of \ref{fig: add} if the added input were quantum, and if the controlling qubit $\ket{\vec{x}}_s$ were instead a qubit in $w$, and if the . Qubits $0$ to $j - 1$ of the $\ket{z}$ register are not shown. }
    \label{fig: squaring}
\end{figure}

\begin{comment}
\begin{figure}[h]
    \centering
    \Qcircuit @C=1em @R=1.6em {
        \lstick{\cdots}          & [w]_0 & & \qquad [w]_1 & & & \qquad [w]_2 & & & [w]_3 & \qquad [w]_2 & & & & [w]_2 & \qquad [w]_1 & & & & [w]_1 & [w]_0 \\
        \lstick{\ket{\vec{x}}_s} & \ctrl{2} & \ctrl{3} & \ctrl{3} & \ctrl{4} & \ctrl{5} & \ctrl{5} & \ctrl{6} & \ctrl{7} & \ctrl{8}
                                 & \ctrl{6} & \ctrl{5} & \ctrl{5} & \ctrl{5} & \ctrl{6} & \ctrl{4} & \ctrl{3} & \ctrl{3} & \ctrl{3} & \ctrl{4} & \ctrl{2} & \qw \\
        \lstick{\cdots}          & \\
        \lstick{\ket{z}_0}       & \ctrl{1} & \qw      & \qw      & \qw      & \qw      & \qw      & \qw      & \qw      & \qw      
                                 & \qw      & \qw      & \qw      & \qw      & \qw      & \qw      & \qw      & \qw      & \qw      & \qw      & \ctrl{1} & \qw \\
        \lstick{\ket{0}}         & \targ    & \ctrl{1} & \ctrl{2} & \qw      & \qw      & \qw      & \qw      & \qw      & \qw 
                                 & \qw      & \qw      & \qw      & \qw      & \qw      & \qw      & \ctrl{2} & \ctrl{1} & \ctrl{1} & \qw      & \targ    & \qw \\
        \lstick{\ket{z}_1}       & \qw      & \ctrl{1} & \qw      & \ctrl{1} & \qw      & \qw      & \qw      & \qw      & \qw 
                                 & \qw      & \qw      & \qw      & \qw      & \qw      & \ctrl{1} & \qw      & \ctrl{1} & \targ    & \targ    & \qw      & \qw \\
        \lstick{\ket{0}}         & \qw      & \targ    & \targ    & \targ    & \ctrl{1} & \ctrl{2} & \qw      & \qw      & \qw
                                 & \qw      & \ctrl{2} & \ctrl{1} & \ctrl{1} & \qw      & \targ    & \targ    & \targ    & \qw      & \qw      & \qw      & \qw \\
        \lstick{\ket{z}_2}       & \qw      & \qw      & \qw      & \qw      & \ctrl{1} & \qw      & \ctrl{1} & \qw      & \qw
                                 & \ctrl{1} & \qw      & \ctrl{1} & \targ    & \targ    & \qw      & \qw      & \qw      & \qw      & \qw      & \qw      & \qw \\
        \lstick{\ket{0}}         & \qw      & \qw      & \qw      & \qw      & \targ    & \targ    & \targ    & \ctrl{1} & \qw
                                 & \targ    & \targ    & \targ    & \qw      & \qw      & \qw      & \qw      & \qw      & \qw      & \qw      & \qw      & \qw \\
        \lstick{\ket{z}_3}       & \qw      & \qw      & \qw      & \qw      & \qw      & \qw      & \qw      & \targ    & \targ
                                 & \qw      & \qw      & \qw      & \qw      & \qw      & \qw      & \qw      & \qw      & \qw      & \qw      & \qw      & \qw
        \gategroup{2}{2}{5}{2}{.7em}{--}
        \gategroup{2}{4}{7}{5}{.7em}{--}
        \gategroup{2}{7}{9}{8}{.7em}{--}
        \gategroup{2}{10}{10}{10}{.7em}{--}
        \gategroup{2}{11}{9}{12}{.7em}{--}
        \gategroup{2}{15}{8}{15}{.7em}{--}
        \gategroup{2}{16}{7}{17}{.7em}{--}
        \gategroup{2}{20}{6}{20}{.7em}{--}
        \gategroup{2}{21}{5}{21}{.7em}{--}
    }
    \caption{Example quantum circuit diagram for a controlled addition of $w$ for $\frac{m}{2} = 4$. The label $[w]_j$ indicates that the correspondingly boxed gates are only applied when $[w]_j = 1$. }
    \label{fig: add}
\end{figure}

\begin{figure}[h]
    \centering
    \Qcircuit @C=0.5em @R=1.2em {
        \lstick{\ket{w}_0} & \ctrl{3} & \qw      & \qw      & \qw      & \qw      & \qw      & \qw      & \qw      & \qw      & \qw      & \qw      & \qw      & \qw      & \qw      
                                 & \qw      & \qw      & \qw      & \qw      & \qw      & \qw      & \qw      & \qw      & \qw      & \qw      & \qw      & \qw      & \qw      & \qw      & \qw      & \qw      & \qw      & \ctrl{3} & \qw \\
        \lstick{\ket{w}_1} & \qw      & \qw      & \ctrl{2} & \ctrl{2} & \qw      & \qw      & \qw      & \qw      & \qw      & \qw      & \qw      & \qw      & \qw      & \qw      
                                 & \qw      & \qw      & \qw      & \qw      & \qw      & \qw      & \qw      & \qw      & \qw      & \qw      & \qw      & \qw      & \ctrl{2} & \ctrl{2} & \qw      & \qw      & \ctrl{2} & \qw      & \qw \\
        \lstick{\cdots}          & & & & & & & & & & & & & & & & & & & & & & & & & & & & & & & & \\
        \lstick{\ket{w}_j} & \ctrl{3} & \ctrl{4} & \ctrl{4} & \ctrl{5} & \qw      & \ctrl{8} & \ctrl{2} & \ctrl{2} & \qw      & \ctrl{12}& \ctrl{2} & \ctrl{2} & \ctrl{14} & \ctrl{2}
                                 & \ctrl{2} & \ctrl{2} & \ctrl{12}& \ctrl{12} & \ctrl{2} & \qw      & \ctrl{2} & \ctrl{2} & \ctrl{8} & \ctrl{8}& \ctrl{2} & \qw      & \ctrl{5} & \ctrl{4} & \ctrl{4} & \ctrl{4} & \ctrl{5} & \ctrl{3} & \qw \\
        \lstick{\cdots}          & & & & & & & & & & & & & & & & & & & & & & & & & & & & & & & & \\
        \lstick{\ket{w}_{\frac{m}{2} - 1}}
                                 & \qw      & \qw      & \qw      & \qw      & \qw      & \qw      & \ctrl{6} & \ctrl{7} & \qw      & \qw      & \ctrl{10}& \ctrl{11}& \qw      & \ctrl{13}
                                 & \ctrl{11}& \ctrl{10}& \qw      & \qw      & \ctrl{11}& \qw      & \ctrl{7} & \ctrl{6} & \qw      & \qw      & \ctrl{7} & \qw      & \qw      & \qw      & \qw      & \qw      & \qw      & \qw      & \qw \\
        \lstick{\ket{z}_j}       & \ctrl{1} & \qw      & \qw      & \qw      & \qw      & \qw      & \qw      & \qw      & \qw      & \qw      & \qw      & \qw      & \qw      & \qw      
                                 & \qw      & \qw      & \qw      & \qw      & \qw      & \qw      & \qw      & \qw      & \qw      & \qw      & \qw      & \qw      & \qw      & \qw      & \qw      & \qw      & \qw      & \ctrl{1} & \qw \\
        \lstick{\ket{0}}         & \targ    & \ctrl{1} & \ctrl{2} & \qw      & \qw      & \qw      & \qw      & \qw      & \qw      & \qw      & \qw      & \qw      & \qw      & \qw 
                                 & \qw      & \qw      & \qw      & \qw      & \qw      & \qw      & \qw      & \qw      & \qw      & \qw      & \qw      & \qw      & \qw      & \ctrl{2} & \ctrl{1} & \ctrl{1} & \qw      & \targ    & \qw \\
        \lstick{\ket{z}_{j + 1}} & \qw      & \ctrl{1} & \qw      & \ctrl{1} & \qw      & \qw      & \qw      & \qw      & \qw      & \qw      & \qw      & \qw      & \qw      & \qw 
                                 & \qw      & \qw      & \qw      & \qw      & \qw      & \qw      & \qw      & \qw      & \qw      & \qw      & \qw      & \qw      & \ctrl{1} & \qw      & \ctrl{1} & \targ    & \targ    & \qw      & \qw \\
        \lstick{\ket{0}}         & \qw      & \targ    & \targ    & \targ    & \qw      & \qw      & \qw      & \qw      & \qw      & \qw      & \qw      & \qw      & \qw      & \qw
                                 & \qw      & \qw      & \qw      & \qw      & \qw      & \qw      & \qw      & \qw      & \qw      & \qw      & \qw      & \qw      & \targ    & \targ    & \targ    & \qw      & \qw      & \qw      & \qw \\
        \lstick{\cdots}          & & & & & \ddots & & & & & & & & & & & & & & & & & & & & & \iddots & & & & & \\
        \lstick{\ket{0}}         & \qw      & \qw      & \qw      & \qw      & \qw      & \ctrl{1} & \ctrl{2} & \qw      & \qw      & \qw      & \qw      & \qw      & \qw      & \qw 
                                 & \qw      & \qw      & \qw      & \qw      & \qw      & \qw      & \qw      & \ctrl{2} & \ctrl{1} & \ctrl{1} & \qw      & \qw      & \qw      & \qw      & \qw      & \qw      & \qw      & \qw      & \qw \\
        \lstick{\ket{z}_{j + \frac{m}{2} - 1}}
                                 & \qw      & \qw      & \qw      & \qw      & \qw      & \ctrl{1} & \qw      & \ctrl{1} & \qw      & \qw      & \qw      & \qw      & \qw      & \qw 
                                 & \qw      & \qw      & \qw      & \qw      & \qw      & \qw      & \ctrl{1} & \qw      & \ctrl{1} & \targ    & \targ    & \qw      & \qw      & \qw      & \qw      & \qw      & \qw      & \qw      & \qw \\
        \lstick{\ket{0}}         & \qw      & \qw      & \qw      & \qw      & \qw      & \targ    & \targ    & \targ    & \qw      & \qw      & \qw      & \qw      & \qw      & \qw
                                 & \qw      & \qw      & \qw      & \qw      & \qw      & \qw      & \targ    & \targ    & \targ    & \qw      & \qw      & \qw      & \qw      & \qw      & \qw      & \qw      & \qw      & \qw      & \qw \\
        \lstick{\cdots}          & & & & & & & & & \ddots & & & & & & & & & & & \iddots & & & & & & & & & & & \\
        \lstick{\ket{0}}         & \qw      & \qw      & \qw      & \qw      & \qw      & \qw      & \qw      & \qw      & \qw      & \ctrl{1} & \ctrl{2} & \qw      & \qw      & \qw
                                 & \qw      & \ctrl{2} & \ctrl{1} & \ctrl{1} & \qw      & \qw      & \qw      & \qw      & \qw      & \qw      & \qw      & \qw      & \qw      & \qw      & \qw      & \qw      & \qw      & \qw      & \qw \\
        \lstick{\ket{z}_{m - 2}} & \qw      & \qw      & \qw      & \qw      & \qw      & \qw      & \qw      & \qw      & \qw      & \ctrl{1} & \qw      & \ctrl{1} & \qw      & \qw
                                 & \ctrl{1} & \qw      & \ctrl{1} & \targ    & \targ    & \qw      & \qw      & \qw      & \qw      & \qw      & \qw      & \qw      & \qw      & \qw      & \qw      & \qw      & \qw      & \qw      & \qw \\
        \lstick{\ket{0}}         & \qw      & \qw      & \qw      & \qw      & \qw      & \qw      & \qw      & \qw      & \qw      & \targ    & \targ    & \targ    & \ctrl{1} & \qw
                                 & \targ    & \targ    & \targ    & \qw      & \qw      & \qw      & \qw      & \qw      & \qw      & \qw      & \qw      & \qw      & \qw      & \qw      & \qw      & \qw      & \qw      & \qw      & \qw \\
        \lstick{\ket{z}_{m - 1}} & \qw      & \qw      & \qw      & \qw      & \qw      & \qw      & \qw      & \qw      & \qw      & \qw      & \qw      & \qw      & \targ    & \targ
                                 & \qw      & \qw      & \qw      & \qw      & \qw      & \qw      & \qw      & \qw      & \qw      & \qw      & \qw      & \qw      & \qw      & \qw      & \qw      & \qw      & \qw      & \qw      & \qw
    }
    \caption{Quantum circuit diagram for $\ket{w}\ket{z} \rightarrow \ket{w}\ket{z + [w]_j2^{M_r - (\frac{m}{2} - 1) + j}w}$. This is essentially the equivalent of \ref{fig: add} if the added input were quantum, and if the controlling qubit $\ket{\vec{x}}_s$ were instead a qubit in $w$, and if the . Qubits $0$ to $j - 1$ of the $\ket{z}$ register are not shown. }
    \label{fig: squaring}
\end{figure}
\end{comment}
%\include{appendixb}

%%% Bibliography (biblatex)  %%%%%%%%%%%%%%%%%%%%%%%%%%%%%%%%%%%%%%%%%%%%%%%%%%%%%%%%%%%%%%%%%%%%%%

\defbibheading{bibintoc}{\chapter*{#1}\addcontentsline{toc}{backmatter}{\refname}} 
% this sets the title of contents name for bibliography to \refname (= References)
% change "backmatter" to "chapter" if you prefer a bold face entry in the table of contents

\printbibliography[title={\refname},heading=bibintoc]

% biblatex also supports chapter-by-chapter bibliography, https://tex.stackexchange.com/a/296502/119566
% see the biblatex manual, section 3.14.3


%%%% Option for natbib %%%%%%%%%%%%%

%%   use an appropriate style (.bst) and your own .bib file[s]

%\bibliographystyle{plainnat}
%\bibliography{mitthesis-sample.bib}

\end{document} 
 