% From mitthesis package
% Version: 1.07, 2024/09/26
% Documentation: https://ctan.org/pkg/mitthesis


\chapter{Conclusion}

In this work, we have designed in detail a quantum circuit that reduces the asymptotic gate complexity of simulating certain types of electronic structure Hamiltonians, in particular those that are low-range. With specific calculations of $T$ counts and error bounds, we showed that asymptotic advantages of the proposed ``double-factorized coherent method'' over the competing ``double-factorized expansion method'' begin to manifest at a fairly low threshold of $n$: for example, fewer than a hundred orbitals for hydrogen chain systems.

Furthermore, we proposed a framework for adapting the usage of the double-factorized coherent method from strictly-low-range electronic structure Hamiltonians to those that are ``low-range dominant'', which tend to be the case for practical systems. We intend to continue formulating more specific decomposition techniques in this framework and numerically analyze the performance of the coherent method in conjunction with qDRIFT and qDRIFT-like methods; unfortunately this was not feasible within the timeframe of this thesis. We strongly encourage this framework as an area of future work for others inspired by the findings in this paper. We also hope that our work opens up a search for other types of electronic structure Hamiltonians (beyond low-range) for which the asymptotic advantages of the coherent method can take effect.

%%%%%%%%%%%%%%%% end table %%%%%%%%%%%%%%%%%%% 

